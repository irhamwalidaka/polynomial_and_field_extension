

\chapter{Perluasan Aljabar}
	\textbf{Definisi 7.1} 	
\par 	Misal $K$ adalah perluasan lapangan dari $F$. $K$ adalah perluasan aljabar dari $F$ jika dan hanya jika, $\forall~\alpha \in K$, $\alpha$ adalah aljabar atas $F$.
\\
\\
\textbf{Contoh}% cari dulu di foto kayaknya.
\\
\\
\textbf{Teorema 7.1}
\par 	Jika $[K:F] < \infty$, maka $K$ adalah perluasan aljabar atas $F$.
\\
\textit{Bukti}
\par 	Karena $[K:F] < \infty$, maka terdapat basis $E= \{ e_1,...,e_n\}$ atas $F$.
\par 	Ambil sembarang $u \in K$, maka akan ada dua kasus, yaitu :
	\begin{enumerate}	
	\renewcommand{\labelenumi}
	{\roman{enumi}}
	\item Jika $u^i=u^j~,~0 \le i<j$, maka $u$ adalah pembuat nol dari $x^i-x^j\in F[x].$
	\item Jika tidak memenuhi i, artinya tidak ada $i~\& ~j,$ dengan $0 \le i<j~$, sedemikian hingga $u^i=u^j$.
\\	Misalkan kita miliki suatu himpunan yang membangun $K$, namun tidak bebas linear, yaitu $\{1_F, u,u^1,u^2,...,u^n\}$. Mengapa tidak bebas linear? Karena banyaknya anggota himpunan tersebut adalah $n+1$, dan dijamin bahwa himpunan yang anggota nya lebih banyak 		dari $n$, di mana $n$ adalah dimensi dari $K$, akan tidak bebas linear (bergantung linear).	
\\	Akibatnya, akan ada $c_i\in F$, tidak semuanya nol, sedemikian sehingga, $$c_01_F + c_1u + c_2u^2+...+c_nu^n = 0.$$
	Jadi, $u$ adalah pembagi nol untuk polinom dengan bentuk, $$c_0+ c_1x + c_2x^2+...+c_nx^n \in F[x].$$	
	\end{enumerate}
\par 	Sehingga, untuk setiap $u\in K,~u~$adalah aljabar atas $F$. Jadi, $K$ adalah perluasan aljabar atas $F$. $\blacksquare$
\par 	Misal $F$ adalah lapangan, $u\in K$ dan $u \notin F$ aljabar atas $F$, maka $F(u):=\{a+bu: a,b\in F\}$ adalah lapangan terkecil yang memuat $F$ dan $u$. Berdasarkan Teorema 6.3, $[F(u):F] = n <\infty$, sehingga, berdasarkan Teorema 7.1, $F(u)$ adalah perluasan aljabar 			atas F.
\\
\par 	Sekarang, coba perhatikan apabila $K$ perluasan lapangan atas $F$, $u_1,...,u_n \in K$ dan $u_1,...,u_n \notin F$ merupakan aljabar atas $F$, maka $F[u_1,...,u_n]$ adalah lapangan terkecil yang memuat $F$ dan $u_1,...,u_n.$ Sehingga, $[F(u_1,...,u_n):F]<\infty$ 			dan  $F(u_1,...,u_n)$ adalah perluasan aljabar atas $F$.
\\
\par 	Bagaimana bisa diperoleh kesimpulan di atas?
\\
\par 	Jadi, untuk memperoleh kesimpulan di atas, dilakukan penyisipan $u_1,...,u_n$ secara bertahap ke lapangan F. Pertama, disisipkan dulu $u_1$ ke dalam $F$, di mana $u_1$ merupakan aljabar atas $F$, sehingga berdasarkan Teorema 6.3, $[F(u_1):F] = <\infty$. Lalu, disisipkan $u_2$ ke dalam $F(u_1)$, di mana $u_2$ merupakan aljabar atas $F(u_1)$, sehingga berdasarkan Teorema 6.3, $[F(u_1,u_2):F(u_1)] = <\infty$. Karena $[F(u_1):F] = <\infty$ dan $[F(u_1,u_2):F(u_1)] = <\infty$, maka,berdasarkan Teorema 6.1, $[F(u_1,u_2):F]<\infty$. Selanjutnya, lakukan penyisipan $u_3,...,u_n$ dengan cara yang sama seperti sebelumnya, dan pada akhirnya diperoleh kesimpulan bahwa $[F(u_1,...,u_n):F]<\infty$ dan  $F(u_1,...,u_n)$ adalah perluasan aljabar atas $F$.
\\
\par 	Perhatikan kembali Teorema 7.1, apakah konvers dari pernyataan tersebut juga benar? (Catatan: konvers dari Teorema 7.1 adalah Jika $K$ adalah perluasan aljabar atas $F$, maka $[K:F] < \infty$.)
\\
\par 	Untuk sembarang $n\ne1 \in \mathbb{N}, 2^{\frac{1}{n}} \notin \mathbb{Q}$. Namun, $2^{\frac{1}{n}}$ adalah pembuat nol untuk $x^n-2 \in \mathbb{Q}[x]$, sehingga $2^{\frac{1}{n}}$ aljabar atas $\mathbb{Q}.$
\\
\par 	Perhatikan bahwa, $x^n-2$ adalah polinom tak tereduksi,berdasarkan Kriteria Eisenstein dengan $p=2$. Selain itu,  $x^n-2$ adalah polinom monik. Akibatnya,  $x^n-2$ adalah polinom minimal untuk $2^{\frac{1}{n}}$. 
\\
\par 	Maka dari itu, berdasarkan Teorema 6.3,berlaku
	\begin{enumerate}
	\item $\mathbb{Q}/((x^n-2)) \cong \mathbb{Q}(2^{\frac{1}{n}}).$ 
	\item $\{1_{\mathbb{Q}},2^{\frac{1}{n}},(2^{\frac{1}{n}})^2,...,(2^{\frac{1}{n}})^{n-1}\}$ adalah basis dari  $\mathbb{Q}(2^{\frac{1}{n}})$ atas $\mathbb{Q}$.
	\item $[ \mathbb{Q}(2^{\frac{1}{n}})~:~\mathbb{Q}]=n<.$
	\end{enumerate}
\par 	Pilih $K=\mathbb{Q}(2^{\frac{1}{n}},2^{\frac{1}{n}},...)$, maka untuk setiap $n\in $
	$\mathbb{N} \setminus \emptyset $, berlaku $n=[ \mathbb{Q}(2^{\frac{1}{n}})~:~\mathbb{Q}] \le [K:\mathbb{Q}.$
	maka $[K:\mathbb{Q}] = \infty.$ 
\\
\par 	Dari bahasan di atas, konvers dari Teorema 7.1 tidak benar.
\\
\\
	\textbf{Teorema 7.2}
\par 	Jika $K=F(u_1,...,u_n)$ adalah perluasan lapangan (berhingga) dari $F$, $u_i$ aljabar atas $F$, $\forall i=1,2,...,n,$ maka $K$ adalah perluasan aljabar dari F (berhingga).
\\
	\textit{Bukti:}
\par 	Pada paragraf sebelum-sebelumnya, kita tahu bahwa $$F\subset F(u_1) \subset F(u_1,u_2)\subset...\subset F(u_1,...,u_n)=K, $$dan karena $u_i$ aljabar atas $F$, maka $u_i$ juga aljabar atas $F(u_1,...,u_n)$. Maka, berdasarkan Teorema 6.3, 
	$$[F(u_1,...,u_{i-1},u_i)~:~F(u_1,...,u_{i-1})] < \infty.$$
\par 	Berdasarkan Teorema 6.1,
	$$\begin{array}{rcl}
	[K:F] &=& [K:F(u_1,...,u_{i-1})][F(u_1,...,u_{i-1}): F(u_1,...,u_{i-2})]...[F(u_1,u_2):F(u_1)]\\
	&~&[F(u_1):F]
	< \infty.
	\end{array}$$
\par 	Jadi, berdasarkan Teorema 7.1, $K$ adalah perluasan aljabar dari $F$.$\blacksquare$
\\
	\textbf{Contoh:}
	\begin{itemize}
	\item $\sqrt{3},\sqrt{5}$ merupakan aljabar atas $\mathbb{Q}$, karena $\sqrt{3}$ merupakan pembuat nol untuk $x^2-3$  dan $\sqrt{5}$ merupakan pembuat nol untuk $x^2-5$, di mana  $x^2-3, x^2-5\in \mathbb{Q}[x]$. Maka, $[\mathbb{Q}(\sqrt{3},\sqrt{5}):			\mathbb{Q}]<\infty$.\\ \\
	Perhatikan bahwa,$\mathbb{Q}(\sqrt{3}) = a+b\sqrt{3}$, maka $\mathbb{Q}(\sqrt{3})=span\{1,\sqrt{3}\}$ dan $[\mathbb{Q}(\sqrt{3}):\mathbb{Q}]=2$. 
	\\ \\
	Bagaimana dengan $[\mathbb{Q}(\sqrt{3},\sqrt{5}):\mathbb{Q}]$? Harus dicari terlebih dahulu polinom minimal dari $\sqrt{5}$ atas $\mathbb{Q}(\sqrt{3})$. Di awal telah kita ketahui bahwa $\sqrt{5}$ merupakan pembuat nol untuk $x^2-5$, karena  $x^2-5$ monik dan tak tereduksi di $\mathbb{Q}(\sqrt{3})[x]$,berdasarkan Kriteria Eisenstein dengan $p=5$, maka $x^2-5$ adalah polinom minimal atas $\mathbb{Q}(\sqrt{3})$.\\ \\
	Andaikan $\pm\sqrt{5} \in \mathbb{Q}(\sqrt{3})$, maka  $\pm\sqrt{5} = a +b\sqrt{3} : a,b\in \mathbb{Q}~,~a,b \ne 0.$ Sehingga,
	$$\begin{array}{rcl}
	(\sqrt{5})^2 &=&  (a +b\sqrt{3})\\
	5&=& a^2+2ab\sqrt{3}+3b^2\\
	\sqrt{3}&=& \frac{5-a^2-3b^2}{2ab} \in\mathbb{Q}.
	\end{array}$$
	Padahal $\sqrt{3} \notin \mathbb{Q}.$ Maka haruslah $\pm\sqrt{5} \notin \mathbb{Q}(\sqrt{3})$.\\ \\
	Maka, berdasarkan Teorema 6.1, $[\mathbb{Q}(\sqrt{3},\sqrt{5}):\mathbb{Q}(\sqrt{3})]=$ deg(polinom minimal)=$2$. Lalu, berdasarkan Teorema 6.1, 
	$$\begin{array}{rcl}
	[\mathbb{Q}(\sqrt{3},\sqrt{5}):\mathbb{Q}] &=& [\mathbb{Q}(\sqrt{3},\sqrt{5}):\mathbb{Q}(\sqrt{3})]\times[\mathbb{Q}(\sqrt{3}):\mathbb{Q}]\\
	&=& 2\times 2\\
	&=& 4.
	\end{array}$$
	\end{itemize}

	\textbf{Teorema 7.3}
\par 	Misalkan $L$ adalah perluasan aljabar atas $K$ dan $K$ adalah perluasan aljabar atas $F$, maka $L$ adalah perluasan aljabar atas $F$.\\
	\textit{Bukti:}
\par 	Karena $L$ adalah perluasan aljabar atas $K$ akan terdapat $u$ suatu aljabar atas $K$, dengan $u\in L$, dan $L$ ruang vektor atas $K$, maka terdapat $a_i \in K$ ,tidak semuanya nol, sedemikian sehingga $a_0+a_1u+...+a_mu^m=0.$
\par 	Karena $K$ adalah perluasan aljabar atas $F$, dengan $K=F(a_1,...,a_m)$, maka $F \subset F(a_1,...,a_m)$ dan $a_1,...,a_m \in F(a_1,...,a_m)$.
\par 	Maka $u$ adalah aljabar atas $F(a_1,...,a_m)$. Berdasarkan Teorema 7.2, $$[F(a_1,...,a_m,u):F(a_1,...,a_m)] < \infty.$$
\par 	Selain itu, $a_i \in F(a_1,...,a_m)$ merupakan aljabar atas $F$ untuk setiap $i=1,2,...,m.$ Sehingga,berdasarkan Teorema 7.2,$[F(a_1,...,a_m) : F]< \infty$.
\par 	Akibatnya, berdasarkan Teorema 6.1, $$[F(a_1,...,a_m,u): F]< \infty.$$ 
	Karenanya,berdasarkan Teorema 7.1,$F(a_1,...,a_m,u)$ adalah perluasan aljabar atas F. Karena $u$ sembarang, maka $L \subseteq F(a_1,...,a_m,u).$ $\blacksquare$
\\
\\
	\textbf{Teorema 7.4}(Hermite)
\par 	$e$ transendental (atas $\mathbb{Q}).$
\\
	\textit{Bukti:}
\par 	Misal $f(x) \in \mathbb{R}$[x] : deg$(f(x))=r. Misalkan juga, F(x)=f(x) + {(1)}(x)+...+f^{(r)}(x).$Akan diperoleh,
	$$\begin{array}{rcl}	
	\frac{d}{dx}(e^{-x}F(x)) &=& -e^{-x}F(x) + e^{-x}(f^{(1)}(x)+...+f^{(r)}(x))\\
	&=&  -e^{-x}f(x).
	\end{array}$$
\par 	Diberikan $k \in \mathbb{N}$, maka $[0,k]$ diperoleh
	$$\begin{array}{rcl}
	\frac{e^{-k}F(k) - F(0)}{k} &=& -e^{-\theta_k k}f(\theta_k k)\\
	F(k) - e^{-k}F(0) &=& -e^{(1-\theta_k) k}f(\theta_k k) k.
	\end{array}$$
\par 	Untuk $k=1,2,...,n$ ,akan diperoleh,
	$$\begin{array}{rcl}
	k=1~~(F(1) - eF(0) &=&  (-1e^{1(1-\theta_1) }f(1\theta_1))~~=\varepsilon_1 \\
	k=2~~(F(2) - eF(0) &=&  (-2e^{2(1-\theta_2) }f(2\theta_2))~~=\varepsilon_2\\
	\vdots~~~~~~~~~~~~~~~~~~~~~~~~ &\vdots& ~~~~~~~~~~~~~~~~~~~~~~~~~~~~\vdots\\
	k=n~~(F(n) - eF(0) &=&  (-ne^{n(1-\theta_n) }f(n\theta_n))~~=\varepsilon_n.
	\end{array}$$
\par 	Misal $e$ adalah aljabar, maka $\exists c_0,c_1,...,c_n \in \mathbb{Z}~~(c_0>0)$ sedemikian sehingga,
	$$\begin{array}{rcl}
	c_0+c_1e+c_2e^2+...+c_ne^n &=&0\\
	c_1e+c_2e^2+...+c_ne^n&=& -c_0
	\end{array}$$
\par 	Selanjutnya dilakukan perkalian $c_k$ dengan $\varepsilon_k$ untuk k yang bersesuaian,
	$$\begin{array}{rcl}
	k=1~~c_1(F(1) - eF(0) &=&  c_1(-1e^{1(1-\theta_1) }f(1\theta_1))~~=c_1\varepsilon_1 \\
	k=2~~c_2(F(2) - eF(0) &=&  c_2(-2e^{2(1-\theta_2) }f(2\theta_2))~~=c_2\varepsilon_2\\
	\vdots~~~~~~~~~~~~~~~~~~~~~~~~~~ &\vdots& ~~~~~~~~~~~~~~~~~~~~~~~~~~~~~~\vdots\\
	k=n~~c_n(F(n) - eF(0) &=&  c_n(-ne^{n(1-\theta_n) }f(n\theta_n))~~=c_n\varepsilon_n.
	\end{array}$$
	Lalu, lakukan penjumlahan dan akan diperoleh,
	$$\begin{array}{rcl}
	&~~& c_1F(1) +c_2F(2) +...+c_nF(n) - F(0)(c_1e+c_2e^2+...+c_ne^n)\\
	&=& c_1\varepsilon_1+c_2\varepsilon_2+...+c_n\varepsilon_n.
	\end{array}$$
\par 	Dengan suatu proses perhitungan, pada akhirnya akan diperoleh
	$$f(x) = \frac{1}{p-1!}x^{p-1}(1-x)^p(2-x)^p...(n-x)^p.$$