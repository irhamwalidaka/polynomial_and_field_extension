\documentclass[12pt,letterpaper]{book}
\linespread{1.3}
\usepackage[indonesian]{babel}
\usepackage{amsmath}
\usepackage{amssymb}

\begin{document}
\frontmatter
\begin{titlepage}     
 \begin{center}
      {\Large\textbf{Polinom dan Aljabar Perluasan   }}\\ % diisi dengan judul skripsi       
\par    \vspace{1cm}           
 	\textbf{\large Catatan Mingguan}     
 \end{center}        
	\vspace{1cm}      
	\hspace{6,8cm}oleh :
\begin{center}Irham Walidaka (1504780)\\ Izdihar Salsabila Noor Arifin (1504406) \\ \end{center} % ganti Nama Penulis dengan nama Anda sendiri
\begin{center}      
	\par \vfill            
	\vspace*{3cm}      {\bf \large Program Studi Matematika}\\ % isi dengan program studi yang Anda ikuti      
	{\bf \large Fakultas Pendidikan Matematika dan Ilmu Pengetahuan Alam (FPMIPA)}\\ % isi dengan nama fakultas tempat Anda kuliah      
	{\bf \large Universitas Pendidikan Indonesia}\\ % isi dengan nama universitas tempat Anda kuliah      
	{\bf  \large 2018 } % sesuaikan tahun dengan tahun kelulusan     
 \end{center} 
\end{titlepage}

\tableofcontents
\mainmatter
\part{Pendahuluan Materi}
\section{Daftar Simbol}
	\begin{itemize}
	\item $\mathbb{Q}$, adalah lapangan bilangan rasional.
	\item $\mathbb{R}$, adalah lapangan bilangan nyata(real).
	\item $\mathbb{C}$, adalah lapangan bilangan kompleks.
	\item $\mathbb{Z}$, adalah lapangan bilangan bulat.
	\item $\mathbb{N}$, adalah lapangan bilangan asli.
	\item $\emptyset$, himpunan kosong.
	\item $\mid$, habis membagi.
	\item $F[x]$, ring polinom atas lapangan F.
	\item $[K,F]$, dimensi ruang vektor $K$ atas lapangan $F$.
	\end{itemize}

\part{Materi Perkuliahan}
\chapter{Ring Polinomial dan Algoritma Pembagian}
	Misal F adalah lapangan, pandang $F[x]$ sebagai himpunan semua suku banyak dengan koefisien elemen di $F$, atau
		$$ F[x]	:= \big\{ a_0 + a_1x + a_2x^2 + ... + a_nx^n | a_i \in F; n \le 0 \big\}. $$
	Untuk sembarang $ p(x), q(x) \in F[x]$ , dapat dinyatakan sebagai
	$$ p(x) = \sum^{n}_{i=0} \ a_ix^i ~~,~~q(x) = \sum^{m}_{i=0} \ b_ix^i   ;~~   n,m \ne 0. $$
\par Dua buah elemen di $F[x]$ dikatakan sama apabila setiap koefisien dari $x^i$ sama untuk setiap $i$, atau $p(x) = q(x)$ jika dan hanya jika $a_i = b_i , \forall i$.
\par 	Didefinisikan operasi penjumlahan pada $F[x]$, sebagai berikut
	$$ +	~~:~~~~F[x] \times F[x] \longrightarrow F[x]$$
	$$ + ~~:(p(x),q(x)) \mapsto p(x) + q(x)$$
	dengan, $$p(x) + q(x) = \sum^{s}_{i=0} \ c_ix^i , di~mana~ c_i = a_i + b_i ~dan~ s =max\{n,m\}$$
	dan $a_{n+1} = a_{n+2} = ... = a_m = 0 $ jika dan hanya jika $m>n$, sebaliknya $b_{m+1} = b_{m+2} = ... = b_n = 0 $ jika dan hanya jika $n>m$.
\par 	Selanjutnya, sebelum didefinisikan operasi perkalian pada $F[x]$, coba perhatikan contoh berikut,
	 $$(a_0+a_1x+a_2x^2) \cdot (b_0+b_1x+b_2x^2+b_3x^3),$$
	koefisien dari $x^4$ dapat diperoleh dari $a_1b_3 + a_2b_2$.  Selain itu,
	$$~~~~(a_0+a_1x+a_2x^2) \cdot (b_0+b_1x+b_2x^2+b_3x^3)$$
	$$=(a_0+a_1x+a_2x^2 + 0x^3+0x^4) \cdot (b_0+b_1x+b_2x^2+b_3x^3+0x^4)$$
	koefisien dari $x^4$ juga dapat diperoleh dari $a_0b_4+a_1b_3+a_2b_2+a_3b_1+a_4b_0$.
	\\ Sehingga, pada $F[x]$ berlaku operasi perkalian , sebagai berikut
	$$ \bullet	~~:~~~~ F[x] \times F[x] \longrightarrow F[x]$$
	$$ \bullet ~~:(p(x),q(x)) \mapsto p(x)\cdot q(x)$$
	dengan, $$p(x) \cdot q(x) = \sum^{t}_{i=0} \ d_ix^i~;~~ t= 2(n+m)$$
	di mana $d_i = a_ib_0 + a_{i-1}b_1+...+a_1b_{i-1}+a_0b_1~~,\forall i$ 
	dan $a_{n+1} = a_{n+2} = ... = a_m = 0$ jika dan hanya jika $m>n$, sebaliknya $b_{m+1} = b_{m+2} = ... = b_n = 0 $ jika dan hanya jika $n>m$.
	
\par 	$(F[x], + , \bullet)$ adalah ring komutatif dengan $1_{F[x]}$. Selanjutnya, ring ini cukup ditulis $F[x]$.
	
\par	Derajat dari $p(x)= \sum^{n}_{i=0} \ a_ix^i $, dinotasikan $\mathrm{deg}(p(x))$, adalah n, dengan n adalah 		pangkat tertinggi dari variabel $x$ pada polinom $p(x)$.
	\\ contoh : 
	\\ $\begin{array}{lcl}
	\mathrm{deg}( x^3 +1) &=& 3
	\\\mathrm{deg} ( a_i) &=& 0 ; a_i \ne 0
	\\\mathrm{deg} ( 0) &=& tidak~ada
	\end{array}$
	\\ \\
	\textbf{ Teorema 1.1}
\par 	Misal $ p(x), q(x) \in F[x]$ dan $ p(x) + q(x) \ne 0$,maka   $\mathrm{deg}(p(x) + q(x)) \le $ 
	\\max $\{\mathrm{deg}(p(x),\mathrm{deg}(q(x))\}$.
	\\
	\textit{Bukti:}
\par 	Untuk membuktikan teorema ini, akan dibagi menjadi dua kasus. Pertama  $\mathrm{deg}(p(x) + q(x)) = \mathrm{max} 
\{\mathrm{deg}(p(x)),\mathrm{deg}(q(x))\}$ dan kedua, $\mathrm{deg}(p(x) + q(x)) <  \mathrm{max} \{\mathrm{deg}(p(x)),\mathrm{deg}(q(x))\}.$
\par	 Perhatikan bahwa, $p(x) = \sum^{n}_{i=0} \ a_ix^i    ,   q(x) = \sum^{m}_{i=0} \ b_ix^i$, maka , $p(x) + q(x) = \sum^{s}_{i=0} \ c_ix^i , di~mana~ c_i = a_i + b_i ~dan~ s =max\{n,m\}$. Akan terdapat dua kemungkinan nilai dari $\mathrm{deg}(p(x) + q(x))$, yaitu bernilai 		$n$ ketika $n>m$ atau bernilai $m$ ketika $m>n$. Akibatnya, $\mathrm{deg}(p(x) + q(x))=\mathrm{max}\{n,m\}$. 
\par 	Misalkan $m=\mathrm{deg}(p(x))$ dan $n=\mathrm{deg}(q(x))$. Dengan memilih $p(x)$ dan $q(x)$ yang berderajat sama, n=m, dan $a_m = -b_n$ , maka $\mathrm{deg}(p(x) + q(x)) < n = m$.
\par 	Jadi, terbukti bahwa  $\mathrm{deg}(p(x) + q(x)) \le $ max $\{\mathrm{deg}(p(x),\mathrm{deg}(q(x))\}$. $\blacksquare$
	\\ 
	\\ \textbf {Teorema 1.2}
\par Misal $p(x),q(x) \ne 0_F \in F[x]$,  maka $\mathrm{\mathrm{deg}}(p(x)q(x))= \mathrm{\mathrm{deg}}(p(x)) + \mathrm{\mathrm{deg}}(q(x))$
\\ 
	\textit{Bukti:}
\par Misalkan $\mathrm{\mathrm{deg}}(p(x))=n$ dan $\mathrm{\mathrm{deg}}(q(x))=m$, dengan $p(x)=a_0+a_1x+...+a_nx^n$ dan $q(x)=b_0+b_1x+...+b_mx^m$, $a_0,...,a_n$ tidak semuanya nol, begitu pula dengan $b_0,...,b_m$. Selanjutnya, perhatikan bahwa 
	$$\begin{array}{rcl}
	\mathrm{\mathrm{deg}}(p(x)q(x))&=& \mathrm{\mathrm{deg}}((a_0+a_1x+...+a_nx^n)(b_0+b_1x+...+b_mx^m))\\
	&=&  \mathrm{\mathrm{deg}}(a_0b_0 + ... + a_nb_mx^{n+m})\\
	&=& n+m\\
	&=& \mathrm{\mathrm{deg}}(p(x))+ \mathrm{\mathrm{deg}}(q(x)).
	\end{array} $$ 
\\
\textbf {Teorema 1.3}
\par $F[x]$ adalah Daerah Integral
\\
\\
	\textit {Bukti:}
\par Ambil $p(x),q(x) \in F[x], p(x),q(x) \ne 0_F, \mathrm{deg}(p(x))= n$ dan $\mathrm{deg} (q(x))= m$.
Perhatikan Bahwa
$p(x)q(x)= a_0b_0 + ... + a_nb_mx^{n+m}$.
Berdasarkan Teorema 1.2 $\mathrm{deg} (p(x)q(x))= n + m \ge 0$.
Jadi, $p(x)q(x)= 0_F$
\\
\\
\textbf {Algoritma Pembagian}
\par Misal $p(x),q(x) \in F[x]$ dengan $m= \mathrm{deg}(q(x)) > 0$, maka $\exists r(x),s(x) \in F[x]$ tunggal sedemikian sehingga $p(x)=q(x)r(x)+s(x)$ di mana $\mathrm{deg}(s(x)) < \mathrm{deg}(q(x))=m$ 
\\
\\ Bukti:
\par Misal $S=\{p(x)-q(x)r(x) | r(x) \in F[x]\}$ dan $s(x) \in S$ sedemikian sehingga  $\forall s_0(x) \in S, \mathrm{deg}(s(x)) \le \mathrm{deg}(s_0(x))$. ($s(x)$ adalah polinom dengan derajat terkecil di $S$). Karena $s(x) \in F(x)$, $s(x)$ dapat ditulis ke dalam bentuk: $$s(x)=\sum ^{t}_{i=0} \ c_ix^i , t>0~~c_t \ne 0_F.$$
\par Asumsikan $t \ge m$. Perhatikan bahwa $s(x)=p(x)-q(x)r(x) \iff s(x)-\frac {c_t}{b_m}x^{t-m}q(x) = p(x)-q(x)r(x) - \frac {c_t}{b_m}x^{t-m}q(x) \iff s(x)- \frac {c_t}{b_m}x^{t-m}q(x) = p(x)-q(x)[r(x)+ \frac {c_t}{b_m}x^{t-m}]$.
\par $Note: Mengapa~kita~menggunakan~\frac {c_t}{b_m}x^{t-m}~?$
\\ $Dengan~menggunakan~kontradiksi,~kita~asumsikan~t \ge m.~Tunjukan~\exists$ $~polinom~\in S~dengan~\mathrm{deg}(polinom)~<~\mathrm{deg}(s(x))~padahal~s(x) adalah$ $~polinom~dengan~derajat~terkecil~di~S.~Maka~
diperoleh~kesimpulan~bahwa$ $~haruslah~t<m$.
\par  Perhatikan bahwa $s(x)=c_0+c_1x+...+c_tx^t$ dan $q(x)=b_0+b_1+...+b_mx^m$. Untuk memperoleh polinom dengan derajat $<t$, maka kita harus mengkontruksi polinom yang menyebabkan $c_txt$ di $s(x))$ hilang. $$q(x)(\frac {c_t}{b_m}x^{t-m})=(b_mx^x+...+b_1x+b_0)(\frac {c_t}{b_m}x^{t-m})$$ $$=c_tx^t+...+\frac {b_1c_t}{b_m}x^{t-m+1}+\frac {b_0c_t}{b_m}x^{t-m}$$
\\ Maka, $\mathrm{deg}(s(x)-\frac {c_t}{b_m}x^{t-m}q(x)) < t$ dengan $s(x)-\frac {c_t}{b_m}x^{t-m} \in S$. 
\\ Hal ini kontradiksi. Jadi, haruslah $t<m$.
\\
\par Bagaimana ketunggalan dari $s(x)$?
\par Misal s(x) dan r(x) itu tidak tunggal. Maka $p(x)=q(s)r_1(x)+s_1(x)$ dan $p(x)=q(x)r_2(x)+s_2(x)$. Akan ditunjukan bahwa, haruslah $s_1(x)=s_2(x)$ dan $r_1(x)=r_2(x)$.
\par Perhatikan Bahwa:
$$p(x)-p(x)=(q(x)r_1(x)+s_1(x))-(q(x)r_2(x)+s_2(x))=0$$
$$0=q(x)(r_1(x)-r_2(x))+(s_1(x)-s_2(x))$$
$$\iff -q(x)(r_1(x)-r_2(x))=s_1(x)-s_2(x)$$
$$\iff q(x)(r_2(x)-r_1(x))=s_1(x)-s_2(x).$$
\\ Karena $\mathrm{deg}(q(x))=m$ dan $\mathrm{deg}(s_1(x)-s_2(x)) \le m$, haruslah \\$r_2(x)-r_1(x)=0_F.$
\par Jadi, $s_1(X)-s_2(x)=0 \to s_1(x)=s_2(x)$ dan $r_1(x)-r_2(x)=0_F \to r_1(x)=r_2(x).$ $\blacksquare$ 
\\ \\
\textbf{Teorema 1.4}
\par $I \ne 0_F$ adalah ideal di $F[x] \to I=\{p(x)q(x)|p(x) \in F[x]\}=(q(x))$
\\
 \textit{Bukti:}
\begin{enumerate}
\item $I \ne 0_F$ maka $\exists g(x) \in I$ sedemikian sehingga $\mathrm{deg}(g(x)) \ge 0$
\item Ambil $q(x) \in I$ di mana $q(x)$ adalah polinomial dengan derajat terkecil \par di $I$, maka haruslah $\mathrm{deg}(q(x)) \le \mathrm{deg}(g(x))$
\item Berdasarkan Algoritma Pembagian
\\ $g(x)=q(x)r(x)+s(x), \exists r(x),s(x) \in F[x]$ di mana $\mathrm{deg}(s(x))<\mathrm{deg}(q(x))$
\item $q(x) \in I$ maka $q(x)r(x) \in I$ akibatnya, untuk $g(x) \in I$ dan $q(x)r(x)$, maka $s(x)=g(x)-q(x)r(x) \in I$
\item Berdasarkan poin 2.
\\ $\mathrm{deg}(q(x) \le \mathrm{deg}(s(x))$
\\ Kasus 1. $\mathrm{deg}(s(x))>0$, (hal ini tidak mungkin)
\\ Kasus 2. $\mathrm{deg}(s(x))=0$, (hal ini tidak mungkin)
\\ Maka haruslah $s(x)=0.$
\end{enumerate}
 $\therefore g(x)=q(x)r(x) \forall g(x) \in I$, dapat ditulis $I=\{r(x)q(x)|r(x) \in F[x]\}.$ $\blacksquare$
\\ \\
\textbf{Definisi 1.1}
\par Misal $R$: Daerah Integral,$R$ adalah daerah integral utama $\iff \forall I \subseteq R~\exists a \in R$, sehingga $I=\{xa|x \in R\}=(a).$
\\ \\
Berdasarkan definisi di atas, Teorema 1.4 dapat dinyatakan sebagai berikut,
\\ \\
\textbf{Teorema 1.4}
\par $F[x]$ adalah daerah ideal utama.
\\ \\
\textbf{Definisi 1.2}
 \par $p(x) \in F[x]$ disebut monik jika dan hanya jika $a_n=1_F$


\chapter{Daerah Ideal Utama}
\textbf{Definisi 2.1}
\par 	Suatu daerah integral ,$R$, disebut sebagai daerah Euclid  jika terdapat fungsi $\delta~:~ R\setminus \{0_R\} \rightarrow \mathbb{N}\cup {0}$, yang memenuhi : 
	\begin{enumerate}
	\item Jika $a,b \in R,$ dan $a,b \ne 0_R$, maka $ \delta(a) \le \delta (ab).$
	\item Jika $a,b \in R,$ dan $b \ne 0_R$, maka $a=bq+r$, dengan $r=0_R$ atau $\delta (r) < \delta(b)$.
	\end{enumerate}
	\textbf{Contoh:}
	\begin{enumerate}
	\item $F[x]$ adalah daerah Euclid dengan fungsi $$\delta~:~ R\setminus \{0_R\} \rightarrow \mathbb{N}\cup {0}$$ $$\delta~:~f(x) \mapsto deg(f(x)).$$
	\item $\mathbb{Z}$ adalah daerah Euclid dengan fungsi $$\delta~:~ \mathbb{Z}\setminus \{0_R\} \rightarrow \mathbb{N}\cup {0}$$ $$\delta~:~z \mapsto |z|.$$
	\item $\mathbb{Z}[\textit{i}] : \{ s + t\textit{i} : s,t \in \mathbb{Z}\}$ adalah daerah Euclid dengan fungsi $$\delta~:~ \mathbb{Z}\setminus \{0_R\} \rightarrow \mathbb{N}\cup {0}$$  $$\delta~:~  s + t\textit{i}~~\mapsto s^2+t^2.$$
	Untuk contoh ini, akan ditunjukan bahwa $\mathbb{Z}[\textit{i}]$ adalah daerah Euclid. Coba perhatikan bahwa
		\begin{itemize}
		\item $\delta$ well-defined. Apabila diambil  $s_1 + t_1\textit{i}~,~ s_2 + t_2\textit{i} \in \mathbb{Z}[\textit{i}]$, dengan $s_1 + t_1\textit{i}~=~ s_2 + t_2\textit{i}$, maka $$\delta(s_1 + t_1\textit{i}) = (s_1)^2 + (t_1)^2 = (s_2)^2 + (t_2)^2= \delta(s_2 +                                                         
		t_2\textit{i}).$$
		\item $\delta$ memenuhi sifat 1 dari Definisi 2.1.  Jika diambil $s+ t\textit{i} \ne 0_{\mathbb{Z}[\textit{i}]}$, maka $\delta (s+ t\textit{i}) = s^2 +t^2 \ge 1.^{(\star)} $ 
\\		Misal $u+ v\textit{i} \ne 0_{\mathbb{Z}[\textit{i}]} \in \mathbb{Z}[\textit{i}]$, maka
			$$\begin{array}{rcl}
			(s+ t\textit{i})(u+ v\textit{i}) &=& su + sv\textit{i} + ut\textit{i} - tv\\
			&=& (su-tv) + (sv+ut)\textit{i}.
			\end{array}$$
		Sehingga, 
			$$\begin{array}{rcl}
			\delta (~(s+ t\textit{i})(u+ v\textit{i})~) &=& (su-tv)^2 + (sv+ut)^2\\
			&=& ((su)^2 +  (tv)^2-2sutv)+ ((tu)^2 +(sv)^2 +2svut)\\
			&=&  (su)^2 + (tu)^2 +(sv)^2 + (tv)^2
			\end{array}$$
		Selain itu,
			$$\begin{array}{rcl}
			\delta ((s+ t\textit{i}))~ \delta((u+ v\textit{i})) &=& (s^2 +t^2)~(u^2 + v^2)\\
			&=& (su)^2 + (tu)^2 +(sv)^2 + (tv)^2
			\end{array}$$
		Maka, kita peroleh bahwa $$\delta ((s+ t\textit{i})(u+ v\textit{i})) = \delta ((s+ t\textit{i}))~ \delta((u+ v\textit{i})).^{(\star \star )}$$
		Jadi, untuk setiap $a,b \in \mathbb{Z}[\textit{i}]$ dengan $a,b \ne 0_{\mathbb{Z}[\textit{i}]}$, berlaku
			$$\begin{array}{rcl}
			\delta (a) &=& \delta(a) \cdot 1\\
			&\le& \delta(a) \delta(b),~berdasarkan ^{(\star)}\\
			&=&  \delta(ab), ~berdasarkan ^{(\star \star)}.
			\end{array}$$
		\item $\delta$ memenuhi sifat 2 dari Definisi 2.1. Jika diambil $a=s + t\textit{i}, b= u + v\textit{i} \in \mathbb{Z}[\textit{i}]$ dengan $a,b \ne 0_{\mathbb{Z}[\textit{i}]}$, maka
			$$\begin{array}{rcl}
			\frac{a}{b}&=& \frac{s + t\textit{i}}{ u + v\textit{i}}\\
			&=& \frac{s + t\textit{i}}{ u + v\textit{i}} \cdot \frac{u - v\textit{i}}{ u - v\textit{i}}\\
			&=& \frac{su + ut\textit{i} + tv - sv\textit{i}}{ u^2 + v^2}\\
			&=& \frac{su+tv}{ u^2 + v^2} + \frac{ut-sv}{ u^2 + v^2} \textit{i}\\
			&=& c+d\textit{i},~~dengan~c,d \in \mathbb{Q}.
			\end{array}$$

%ket:ada gambar garis bilangan

		Selanjutnya, akan terdapat $m,n\in \mathbb{Z}$,sedemikian sehingga $|m-c| \le \frac{1}{2}$ dan $|n-d| \le \frac{1}{2}$. Akibatnya,
			$$\begin{array}{rcl}
			\frac{a}{b} &=& c+d\textit{i}\\
			a &=& b (c+d\textit{i})\\
			&=& b((c-m+m) + (d+n-n) \textit{i})\\
			&=& b((m+n\textit{i})+((c-m) +(d-n)\textit{i})).
			\end{array}$$
		Lalu, akan terdapat $p,q \in \mathbb{Z}[\textit{i}]$, sedemikian sehingga untuk $a,b\in \mathbb{Z}$ yang sudah diberikan sebelumnya, diperoleh $p = b(m+n\textit{i})$ dan $q= b((c-m) +(d-n)\textit{i}).$
\\		Apakah $\delta(b) > \delta(q)$? Coba perhatikan,
			$$\begin{array}{rcl}
			\delta(bq) &=& \delta(b) \delta(q)\\
			&=& \delta(b)((c-m)^2+(d-n)^2)\\
			&\le& \delta(b)(\frac{1}{4} +\frac{1}{4)}\\
			&=& \frac{1}{2} \delta(b)
			\end{array}$$
%Lanjutan masih di foto
		\end{itemize}
	\end{enumerate}
	$\blacksquare$
\\
\textbf{Definisi 2.2}
\par 	Daerah Integral ,$R$, yang setiap idealnya memiliki bentuk $(c):=\{ rc~:~r\in R\}$, untuk suatu $c\in R$, disebut Daerah Ideal Utama.
\\
\\	\textbf{Contoh:}
	\begin{itemize}
	\item $\mathbb{Z}$ adalah Daerah Ideal Utama.
	\begin{enumerate}
	\item Jika $I =(0)$, maka $I=\{z0: z\in \mathbb{Z}\}$.
	\item Jika $I \ne (0)$, misalkan $a>0 \in I$ sedemikian sehingga untuk setiap $b \in I$, berlaku $a \le b$. Selanjutnya, pada $\mathbb{Z}$ berlaku algoritma pembagian, sehingga $$b=aq+r~~,~\exists q,r \in \mathbb{Z} ~\&~ 0\le r <a.$$
	Perhatikan bahwa, $r=b-aq \in I.$ Karena $a$ adalah elemen terkecil di $I$, namun ada elemen lain,$r$, yang lebih kecil dari $a$,maka haruslah r=0.Akibatnya $b=aq$ dan $b\in (a).$
	\end{enumerate}
	Dari 1 dan 2, dapat disimpulkan bahwa setiap ideal dari $\mathbb{Z}$ memiliki bentuk $(a)$, sehingga $\mathbb{Z}$ adalah Daerah Ideal Utama.
	\end{itemize}
\textbf{Teorema 2.1}
\par 	Setiap daerah Euclid adalah Daerah Ideal Utama.
\\
	\textit{Bukti:}
\par 	Ambil sembarang ideal di daerah Euclid R, yaitu  $I \ne (0)$. Maka, $$ \Phi := \{\delta(i)~:~i\in I\} \subset \mathbb{N} \cup \{0\} $$, dan $\Phi$ tak kosong.
\par 	Berdasarkan sifat keterurutan baik (\textit{well-ordering}), maka terdapat elemen terkecil di $\Phi$, sama artinya dengan,  terdapat $b \in I$ sedemikian sehingga $\delta(b)\le \delta(i), \forall i \in I.$
\par 	Selanjutnya, jika diambil sembarang $a \in I$, maka $a=bq+r~~,~\exists q,r \in R$ dan nilai $r$ yang mungkin adalah $r=0_R$ atau $\delta(r) < \delta(b).$
\par 	Karena  $r=a-bq \in I$, tidaklah mungkin $\delta(r) < \delta(b)$, maka haruslah $r=0_R$, sehingga $a=bq\in (b).$ Sehingga $I \subseteq (b). $
\par 	Untuk setiap $r \in R$, $rb \in (b),$ sehingga $rb \in I.$ Sehingga, $(b)\subseteq I$.
\par 	Jadi, $I=(b)$. Karena setiap ideal di daerah Euclid memiliki bentuk $(b)$, maka daerah Euclid merupakan Daerah Ideal Utama.$\blacksquare$
\\
\\
\textbf{Definisi 2.3}
\par 	Daerah Integral R memenuhi Kondisi Rantai Naik (KRN), jika $(a_i) \subseteq (a_2)\subseteq (a_3) \subseteq ...$~, maka terdapat $n \in \mathbb{N}$ sedemikian sehingga $(a_i) \subseteq (a_n)~,~ \forall i \ge n.$ 
\\
\\
\textbf{Contoh},Pada $\mathbb{Z}$,
	$$(81)\subseteq (9) \subseteq (3) \subseteq (1) = \mathbb{Z} = (1) = (1) = ...~.$$
\textbf{Bukan Contoh}, Pada $\mathbb{Q}_{\mathbb{Z}}[x],$ dengan
	$$\mathbb{Q}_{\mathbb{Z}}[x] := \{ a_0+a_1x+a_2x^2+... : a_0 \in \mathbb{Z}, a_i\in \mathbb{Q} \forall i > 0 \}.$$



\chapter{Daerah Faktorisasi Tunggal}
\textbf{Teorema 3.1}
\par 	Misal $F$ adalah lapangan. Maka $f(x)$ adalah unit di $F[x]$ jika dan hanya jika $f(x)$ adalah polinom konstanta yang tidak nol.
\\
	\textit{Bukti:}
\\ $(\Rightarrow)$
\par 	Misal $f(x)$ adalah unit di $F[x]$, $F$ lapangan. Akan ditunjukan $f(x)$ adalah polinom konstanta yang tak nol. $f(x)$ unit, sehingga $f(x)g(x)=1_F, ~ \exists g(x) \in F[x].$
\par 	Perhatikan Bahwa, $$deg(f(x)g(x))=deg(1_F) \iff deg(f(x))+deg(g(x))=0,$$  Maka, haruslah $deg(f(x))=0$ dan deg$(g(x))=0.$ Jadi, $f(x)$ dan $g(x)$ adalah polinom konstanta yang tak nol.
\\ $(\Leftarrow)$
\par 	Ambil $f(x)=b \in F[x]~b \ne 0.$ Karena $b \in F$, maka $\exists b^{-1} \in F$ sedemikian sehingga $bb^{-1}=1_F.$ 
\par 	Dengan asumsi $g(x)=b^{-1} \in F[x]$, kita peroleh $f(x)g(x)=bb^{-1}=1_F.$ Jadi, $f(x)$ adalah unit di $F[x].$ $\blacksquare$
\\	
\\
	\textbf{Catatan}
\begin{itemize}
\item Suatu polinomial $f(x) \in F[x]$ disebut $associate$ dari $g(x) \in F[x]$ jika $f(x)=cg(x)$ untuk suatu $c \in F, c \ne 0.$
\item $f(x)$ adalah $associate$ dari $g(x) \iff g(x)~associate$ dari $f(x).$ 
\end{itemize}
\textbf {Definisi 3.1}
\par Misal $F$ adalah lapangan. Suatu polinom tak konstan $p(x) \in F[x]$ dikatakan tak tereduksi jika pembaginya hanyalah $associate$-nya dan polinom konstan yang tak nol (unit). Polinom tak konstan yang tidak "tidak tereduksi" dikatakan tereduksi.
\\
\\ Contoh:
\par $x+2$ tak tereduksi di $Q[x]$, karena berdasarkan Algoritma Pembagian, pembagi dari $x+2$ haruslah berderajat 1 atau 0.
\par 	Misal $f(x)|(x+2)$ maka $x+2=f(x)g(x)$ untuk suatu $f(x),g(x) \in Q[x].$ Jika $deg(f(x))=1$ maka $deg(g(x))=0$ sehingga $g(x)=c \in Q$. Akibatnya $c^{-1}(x+2)=f(x)$ dan $f(x)$ adalah $associate$ dari $x+2$.
\\
\\
\textbf{Teorema 3.2}
\par Misal $F$ adalah lapangan dan $p(x) \in F[x]$ dengan $deg(p(x))>0$
\\ Maka kondisi dibawah ini ekuivalen:
\begin{enumerate}
\item $p(x)$ tak tereduksi.
\item Jika $b(x)$ dan $c(x)$ adalah sembarang polinom sedemikian sehingga $p(x)|b(x)c(x)$, maka $p(x)|b(x)$ atau $p(x)|c(x)$
\item Jika $r(x)$ dan $s(x)$ adalah sembarang polinom sedemikian sehingga $p(x)=r(x)s(x)$, maka $r(x)$ atau $s(x)$ adalah polinom konstan tak nol.
\end{enumerate}
	\textit{Bukti:}
\\ $(1.)\Rightarrow(2.)$
\par 	Misalkan $p(x)$ tak tereduksi dan $\exists b(x),c(x) \in F[x]$ sedemikian sehingga $p(x)|b(x)c(x).$
\par 	Akan ditunjukan $p(x)|b(x)$ atau $p(x)|c(x).$ 
\par 	Perhatikan $fpb(p(x),b(x))$, maka haruslah $fpb$ dari $p(x)$ dan $b(x)$ adalah pembagi dari $p(x)$ yang tak nol, dan karena $p(x)$ tak tereduksi, maka pembagi $p(x)$ yang mungkin adalah $associate$-nya atau unit. Jika $fpb(p(x),b(x))$ adalah $associate$ dari $p(x)$, maka $p(x)|b(x)$. Jika $fpb(p(x),b(x))$ adalah unit, maka $p(x)|c(x).$
\\
\\ $(2.)\Rightarrow(3.)$
\par 	Misal $p(x)=r(x)s(x)$ maka $p(x)|r(x)$ atau $p(x)|s(x)$, (berdasarkan poin 2).
\par 	Jika $p(x)|r(x)$, dengan $r(x)=p(x)v(x)$, maka, 
	$$ p(x)=r(x)s(x) \iff  p(x)=p(x)v(x)s(x) \iff  v(x)s(x)=1_F.$$
	Akibatnya $s(x)$ adalah unit. Jadi $s(x)$ adalah konstanta tak nol. Begitu pula $p(x)|s(x).$
\\
\\ $(3.)\Rightarrow(1.)$
\par Misal $c(x)$ adalah sembarang pembagi dari $p(x)$, maka $\exists d(x) \in F[x]$ sedemikian sehingga $p(x)=c(x)d(x).$ $c(x),d(x)$ adalah konstanta tak nol (berdasarkan poin 3).
\par 	Jika $d(x)=\alpha \ne 0_F$ maka,
	$$ p(x)=c(x)\alpha \iff  \alpha ^{-1}p(x)=\alpha ^{-1}c(x)\alpha \iff  \alpha ^{-1}p(x)=c(x).$$
Akibatnya $c(x)$ adalah $associate$ dari $p(x)$ dan konstanta tak nol. Jadi $p(x)$ tak tereduksi.$\blacksquare$
\\
\\
\textbf {Corollary 3.3}
\par Misal $F$ adalah lapangan dan $p(x) \in F[x]$ adalah polinom tak tereduksi. Jika $p(x)|a_1(x)a_2(x)...a_n(x)$, maka $p(x)$ membagi setidaknya satu $a_i(x).$
\\
	\textit{Bukti:}
\par Jika $p(x)|a_1(x)a_2(x)...a_n(x)$, maka $p(x)|a_1(x)$ atau $p(x)|a_2(x)...a_n(x)$, \\ berdasarkan Teorema 2.1(2).
	\begin{itemize}
	\item Kasus 1: Jika $p(x)|a_1(x)$ maka terbukti bahwa $p(x)$ membagi setidaknya satu $a_i(x)$ yang dalam konteks ini $a_1(x)$
	\item Kasus 2: Jika $p(x)|a_2(x)...a_n(x)$, maka $p(x)|a_2(x)$ atau $p(x)|a_3(x)...a_n(x)$ berdasarkan 2.1(2). Sampai dengan n kali, terbukti bahwa $p(x)$ membagi setidaknya satu $a_i(x)$
	\end{itemize}$\blacksquare$
\\
\\
\textbf{Teorema 3.4}
\par Misal $F$ adalah lapangan, setiap polinom tak konstan $f(x) \in F[x]$ adalah produk dari polinom-polinom tak tereduksi di $F[x]$ dan faktorisasinya bersifat unik. Dengan kata lain, jika $f(x)=p_1(x)p_2(x)...p_r(x)$ dan $f(x)=q_1(x)q_2(x)...q_s(x)$ dengan $p_i(x)$ dan $q_j(x)$ polinom tak tereduksi, maka $r=s$, (banyak faktornya sama). Setelah $q_j(x)$ diurutkan kembali dan ditandai ulang jika perlu, maka $p_i(x)$ adalah $associate$ dari $q_i(x)~(i=1,2,...,r)$



\chapter{Lemma Gauss dan Kriteria Eisenstein}
\par 	Misal $R$ adalah daerah faktorisasi tunggal. $p(x) \ne 0_R \in R[x]$ disebut primitif, jika dan hanya jika setiap konstanta $c$ di $R[x]$ yang habis membagi $p(x)$ , dinotasikan $c|p(x)$, merupakan unit.
	\\
	\textbf{ Contoh :}
	 \begin{enumerate}
	\item Ambil $6x^2 + 18x+12 \in Z[x]$
	\\
	Perhatikan bahwa\\
	$\begin{array} {rcl}
	6x^2 + 18x+12 &=& \pm 1 \cdot \pm(6x^2 + 18x+12)\\
	&=& \pm 2 \cdot \pm(3x^2 + 9x+6)\\
	&=& \pm 3 \cdot \pm(2x^2 + 6x+4)\\
	&=& \pm 6 \cdot \pm(1x^2 + 3x+2)
	\end{array}$\\
	Elemen unit pada  $Z[x]$ hanyalah $\pm1$, namun $6x^2 + 18x+12$ juga habis dibagi oleh konstanta $\pm 2, \pm 3$ dan $ \pm 6$ yang bukan unit. Jadi,  $6x^2 + 18x+12$ bukan elemen primitif di $Z[x]$.
	\item Ambil $3x^2 + -5x+2 \in Z[x]$
	\\
	Perhatikan bahwa
	\\
	$3x^2 - 5x+ 2 = \pm 1 \cdot \pm(3x^2 - 5x+2)$\\
	Karena $3x^2  -5x+2 $ hanya habis dibagi oleh $\pm1$, yang merupakan unit di $Z[x]$, maka $3x^2  -5x+2 $ merupakan elemen primitif.
	\end{enumerate}
	\textbf{ Lemma 4.1}
\par 	Jika $p(x) \in R[x]$ primitif dan deg$(p(x))=1$, maka $p(x)$ tak tereduksi. 
	\\
	\textit{Bukti:}
\par 	Karena $p(x)$ primitif dan deg$(p(x)) =1$ , maka bentuk $p(x)=u \cdot q(x)$ akan selalu ada dengan u adalah konstanta di $R[x]$ yang juga unit dan $q(x)$ adalah polinom berderajat 1 yang merupakan \textit{associate} dari $p(x)$. Jadi, $p(x)$ tak tereduksi.$\blacksquare$
	\\
	\\
	\textbf{Catatan :}
\par 	Jika deg$(p(x)) > 1$, maka tidak bisa ditarik kesimpulan apapun. Misal, ambil dua buah polinom primitif di $Z[x]$ yang berderajat dua, yaitu $x^2 + 1$ dan $x^2 + 3x +2$.  $x^2 + 1$ merupakan polinom tak tereduksi di $Z[x]$, karena hanya habis dibagi oleh 		unit di $Z$ yaitu $\pm1$ dan \textit{associate} nya.  $x^2 + 3x +2$ juga habis dibagi oleh unit di $Z$ yaitu $\pm1$ dan \textit{associate} nya,namun, $x^2 + 3x +2$ juga bisa dinyatakan sebagai perkalian dari $(x+2)(x+1)$. Sehingga, $x^2 + 3x +2$ 			tereduksi.
	\\
	\\
	\textbf{Lemma 4.2}
\par 	Jika $p(x)\in R[x]$ adalah polinom tak terduksi dan deg$(p(x)) > 0$, maka $p(x)$ primitif.
	\\
	\\
	\textbf{Lemma Gauss}
\par 	Jika $g(x)~,~h(x)\in R[x]$ , maka $(g\cdot h)(x)$ primitif.
	\\
	\textit{Bukti:}
\par 	Untuk membuktikan Lemma Gauss, akan digunakan kontradiksi. Asumsikan $f(x) = g(x)\cdot h(x)$ tidak primitif, artinya $ \exists ~c \in R[x]$ dan $c | f(x)$ sedemikian sehingga $c$ bukan unit.
\par 	Perhatikan bahwa $c \ne 0_R$. Karena $R$ adalah daerah faktorisasi tunggal, maka $\exists p_1,...,p_r$ bilangan prima sedemikian sehingga $c = p_1\cdot p_2 \cdot ...\cdot p_r$. Akibatnya, $p_i|f(x)~~\forall i \in \{1,...,r\}$.
\par 	Akan dibuktikan $p_i|g(x)~\vee~ p_i|h(x)~~\forall i \in \{1,...,r\}$. (Catatan : Jika pernyataan ini benar, maka pembagi konstan untuk $g(x)~~\vee~~h(x)$ ada yang bukan unit. Kontradiksi dengan fakta bahwa $g(x),h(x)$ primitif.). 
\par 	$f(x) = ~g(x) \cdot h(x)~ \iff \sum^{m+n}_{i=0} \ a_ix^i~=~\sum^{m}_{i=0} \ b_ix^i \cdot \sum^{n}_{i=0} \ c_ix^i$, $a_i,b_i,c_i\in R~,\forall i$.
\par 	Ambil sembarang $0\le i \le r$. Jika $p_i |a_j~~\forall~j\in \{0,1,...,m+n\}$ , maka $p_i |b_j~~\forall~j\in\{0,1,...,m\}$ atau $p_i |c_j~~\forall~j\in \{0,1,...,n\}$. Akibatnya, berdasarkan Teorema 4.3, $p_i|g(x)~\vee~ p_i|h(x)$. Hal ini kontradiksi dengan fakta bahwa $g(x)$ \& $h(x)$ primitif. Jadi , haruslah $f(x) = ~g(x) \cdot h(x)~$ primitif.$\blacksquare$
	\\
	\\
	\textbf{Teorema 4.3}
\par 	Misal $m,n \in \mathbb{N}$, $p \in R$ tak tereduksi, dengan $R$ daerah faktorisasi tunggal. $\sum^{m+n}_{i=0} \ a_ix^i~,~\sum^{m}_{i=0} \ b_ix^i~,~ \sum^{n}_{i=0} \ c_ix^i \in R[x]$, dengan $$\sum^{m+n}_{i=0} \ a_ix^i~=~\sum^{m}_{i=0} \ b_ix^i \cdot \sum^{n}_{i=0} \ c_ix^i.$$
\par 	Jika $p |a_j~~\forall~j\in \{0,1,...,m+n\}$ , maka $p |b_j~~\forall~j\in\{0,1,...,m\}$ atau $p |c_j~~\forall~j\in \{0,1,...,n\}$.
	\\
	\textit{Bukti:}
\par 	Untuk membuktikan Teorema ini, akan digunakan kontradiksi. Asumsikan $\exists r\in \{0,1,...,m\}$ dan $t \in \{0,1,...,n\}$ sedemikian sehingga $p$ tidak membagi $b_r$ dan $p$ tidak membagi $c_t$, namun, $p|b_i~\forall i\in \{0,1,...,r-1\}$ dan $p|c_i~\forall i\in 			\{0,1,...,t-1\}$. Dengan kata lain, dipilih indeks ($r$ dan $t$) terkecil di mana $p$ tidak habis membagi indeks tersebut, sehingga bisa dijamin bahwa indeks yang lebih kecil dari $r$ habis dibagi oleh $p$ ,begitu pula dengan $t$. 
\par 	Karena $p$ tidak habis membagi $b_r$ dan $c_t$ , maka haruslah $p$ juga tidak habis membagi $b_rc_t$.
\par 	Perhatikan bahwa, nilai koefisien dari variabel $x^{r+t}$, yaitu $a_{r+t}$, dapat diperoleh sebagai berikut
	$$ a_{r+t}~=~b_0c_{r+t}+b_1c_{r+t-1}+...+b_{r-1}c_{t+1}+b_rc_t+b_{r+1}c_{t-1}+...+b_{r+t-1}c_1+b_{r+t}c_0$$
\par 	Dari persamaan di atas, kita peroleh persamaan untuk $b_rc_t$, yaitu
	$$\begin{array}{rcl}
	 b_rc_t&=&a_{r+t} - ((b_0c_{r+t}+b_1c_{r+t-1}+...+b_{r-1}c_{t+1})+(b_{r+1}c_{t-1}+...+\\
	&& b_{r+t-1}c_1+b_{r+t}c_0))
	\end{array}$$
	Pada pernyataan awal, diketahui bahwa $p |a_j~~\forall~j\in \{0,1,...,m+n\}$ maka $p|a_{r+t}$. Karena $p|b_i~\forall i\in \{0,1,...,r-1\}$ , maka $p|(b_0c_{r+t}+b_1c_{r+t-1}+...+b_{r-1}c_{t+1})$. Selanjutnya, karena $p|c_i~\forall i\in 						\{0,1,...,t-1\}$, maka $p|(b_{r+1}c_{t-1}+...+b_{r+t-1}c_1+b_{r+t}c_0)$. Akibatnya, $p|b_rc_t$. Hal ini kontradiksi. Jadi, haruslah $p|b_r$ atau $p|c_t$.
\par 	Kesimpulannya, $p |b_j~~\forall~j\in\{0,1,...,m\}$ atau $p |c_j~~\forall~j\in \{0,1,...,n\}$. $\blacksquare$
	\\
	\\
	\textbf{Kriteria Eisenstein}
\par 	Misal $f(x)= \sum^{n}_{i=0} \ a_ix^i \in \mathbb{Z}[x]$  dan deg$(f(x)) >0$. Terdapat bilangan prima, $p$, sedemikian sehingga $p|a_i~,~\forall i \in \{0,1,...,n-1\}$. Jika $p|a_n$ dan $p^2|a_0$,  maka $f(x)$ tak tereduksi di $\mathbb{Q}[x]$.
	\\
	\textit{Bukti:}
\par 	Dengan menggunakan kontradiksi, akan dibuktikan kebenaran dari teorema ini. Misal $f(x)= \sum^{n}_{i=0} \ a_ix^i \in \mathbb{Z}[x]$. Asumsikan $f(x)$ tereduksi di $\mathbb{Q}[x]$, artinya $f(x) = g(x) \cdot h(x)$ dengan $g(x), h(x) \in \mathbb{Q}[x]$ yang tereduksi. 
\par 	Selanjutnya, perhatikan bahwa $\exists~c,d \in \mathbb{Z}$ sedemikian sehingga $c\cdot g(x),d\cdot h(x) \in \mathbb{Z}[x]$. Akibatnya, $cd \cdot f(x) = (c\cdot g(x))(d\cdot h(x)) \in \mathbb {Z}[x]$. (Catatan : $c$ dan $d$ masing-masing adalah KPK dari penyebut 			koefisien-koefisien rasional dari $g(x)$ dan $h(x)$).
\par 	Misal $p$ adalah suatu bilangan prima, sedemikian sehingga $p|cd$. Karena $cd \ne 0$ dan $cd\in \mathbb{Z}$ bukan unit ,dengan $\mathbb{Z}$ merupakan daerah faktorisasi tunggal, maka dapat dinyatakan $$cd = p\cdot p_1\cdot p_2\cdot ... \cdot p_r.$$ Selain 		itu, $cd~=~pt$, artinya $p| (cda_i)$ , $\forall i \in \{0,1,...,n\}$. Maka,  $p| (cb_i)$ , $\forall i \in \{0,1,...,n\}$ atau  $p| (dc_i)$ , $\forall i \in \{0,1,...,n\}$. 
\par 	Pilih kondisi ketika $p| (cb_i)$ , $\forall i \in \{0,1,...,n\}$, maka $$c \cdot g(x) = p \cdot k(x) $$ untuk suatu $k(x) \in \mathbb{Z}[x]$. Akibatnya, 
	$$\begin{array}{rcl}
	pt \cdot f(x) &=& cd\cdot f(x)\\
	\iff pt \cdot f(x) &=& (c\cdot g(x)) \cdot (d\cdot h(x))\\
	\iff pt \cdot f(x) &=& (p\cdot k(x)) \cdot (d\cdot h(x))\\
	\iff ~t \cdot f(x) &=&(g(x)) \cdot (d\cdot h(x))
	\end{array}$$
\par 	Sekarang pilih $q$ ,suatu bilangan prima, sedemikian sehingga $q|t$ atau sama artinya dengan $t = qs$ untuk suatu $s \in \mathbb{Z}$. Karenanya, jika $q$ habis membagi semua koefisien dari $t\cdot f(x)$ , maka $q$ juga habis membagi $(1\cdot k(x)$ atau $			(d \cdot h(x))$. Karena $q$ tidak mungkin habis membagi 1 dan tidak ada jaminan bahwa $q$ akan habis membagi semua koefisien dari $k(x)$, maka haruslah $q$ habis membagi $(d\cdot h(x))$, artinya $d\cdot h(x) = q\cdot l(x)$ untuk suatu $l(x)\in \mathbb{Z}[x].$ 			Sehingga,
	$$\begin{array}{rcl}
	t \cdot f(x) &=&(g(x)) \cdot (d\cdot h(x))\\
	\iff qs\cdot f(x) &=&  k(x)\cdot (q\cdot l(x))\\
	\iff ~~s\cdot f(x) &=& k(x) \cdot l(x).
	\end{array}$$
\par 	Jika $s$ unit, maka $f(x)$ adalah polinom tereduksi di $\mathbb{Z}[x]$.
\par 	Jika $s$ bukan unit dan $s\ne 0$ , maka $s=p_1p_2...p_n$. Sehingga, $s\cdot f(x) = p_1p_2...p_n\cdot f(x) = k(x)l(x)$. Jika $p_1|s$, maka $p_1|s\cdot f(x) = k(x)l(x)$, artinya $p_1|k(x)$ atau $p_1|l(x)$.
\par 	Misal $k(x) = p_1\cdot r(x)$, sehingga
	$$\begin{array}{rcl}
	p_1p_2...p_n \cdot f(x) &=& p_1\cdot r(x)l(x)\\
	\iff p_2p_3...p_n \cdot f(x) &=& r(x)l(x).
	\end{array}$$
\par 	Dengan menggunakan sebanyak n-1 langkah yang sama dengan sebelumnya, maka akan diperoleh $f(x)=\phi (x)\psi (x)$ untuk suatu $\phi(x),\psi(x) \in \mathbb{Z}[x]$. Artinya $f(x)\in \mathbb{Q}[x]$ tereduksi di $\mathbb{Z}[x]$.
\par 	Karena f(x) tereduksi di $\mathbb{Z}[x]$, maka
	$$ f(x) = \sum^{n}_{i=0} \ a_ix^i~=~\sum^{r}_{i=0} \ b_ix^i \cdot \sum^{s}_{i=0} \ c_ix^i$$
	dengan $n=r+s$ , $b_i,c_j \in \mathbb{Z}$, dan $r,s \ge 1.$
\par 	Berdasarkan fakta yang dimiliki, $p|a_0$ dengan $a_0 = b_0c_0$, maka $p|b_0$ atau $p|c_0$. Jika $p|c_0$ dan $(p\cdot p)$ tidak habis membagi $a_0=$, maka $p$ jelas tidak habis membagi $b_0.$
\par 	Fakta yang lain menyebutkan bahwa, $p$ tidak habis membagi $a_n = b_rc_s$, maka $p$ tidak habis membagi $b_r$ dan $c_s$.
\par 	Misalkan $k\in \{0,1,...,s-1\}$ sedemikian sehingga $p$ tidak habis membagi $c_k$, namun $p|c_i~\forall i \in \{0,1,...,k-1\}.$
\par 	Sekarang, perhatikan bahwa
	$$\begin{array}{rcl}
	a_k &=& b_0c_k+b_1c_{k-1}+...+b_{k-1}c_1+b_kc_0\\
	\iff b_0c_k &=& a_k - (b_1c_{k-1}+...+b_{k-1}c_1+b_kc_0).
	\end{array}$$
	Dari persamaan di atas, diperoleh kesimpulan bahwa $p|b_0c_k$. Mengapa? Karena $p|a_k$ dan $p|(b_1c_{k-1}+...+b_{k-1}c_1+b_kc_0)$.
\par 	Karena $p|b_0c_k$, maka $p|b_0$ atau $p|c_k$. Dikarenakan $p$ tidak habis membagi $c_k$, maka haruslah $p|b_0$. Namun hal ini kontradiksi dengan fakta bahwa $p$ tidak habis membagi $b_0$. Jadi, haruslah $f(x)$ tak tereduksi di $\mathbb{Q}[x]$. $\blacksquare$
\\
\chapter{Peluasan Lapangan}
\par 	Misal $f(x),g(x),p(x) \in F[x]$, dengan $p(x) \ne 0$. Polinom $f(x)$ dan $g(x)$ dikatakan memiliki relasi terhadap $p(x)$, ditandai dengan $f(x) \equiv g(x)~mod~(p(x))$, jika $p(x)|f(x) - g(x)$. Sifat dari relasi "$\equiv$" , dinyatakan dalam teorema berikut.
\\
\\
	\textbf{Teorema 5.1}
\par 	Relasi di $F[x]$ terhadap polinom $p(x)$, yaitu "$\equiv$", merupakan relasi ekuivalen. Sehingga memenuhi sifat:
	\begin{enumerate}
	\item Reflektif, $f(x)\equiv f(x)~mod~(p(x))$.
	\item Simetri, Jika $f(x)\equiv g(x)~mod~(p(x))$, maka $g(x)\equiv f(x)~mod~(p(x))$.
	\item Transitif, Jika $f(x)\equiv g(x)~mod~(p(x))$ dan $g(x)\equiv h(x)~mod~(p(x))$, maka $f(x)\equiv h(x)~mod~(p(x))$.
	\end{enumerate}
	\textit{Bukti}
\par 	Ambil sembarang $f(x),g(x),p(x) \in F[x]$, dengan $p(x) \ne 0$. Akan ditunjukan relasi "$\equiv$" di $F[x]$ terhadap $p(x)$ memenuhi sifat reflektif,simetri dan transitif.
	\begin{enumerate}
	\item Perhatikan bahwa, $p(x)|(f(x)-f(x)) = 0$, maka $f(x)\equiv f(x)~mod~(p(x))$. Jadi sifat reflektif terpenuhi.
	\item Karena $f(x)\equiv g(x)~mod~(p(x))$, maka $p(x)| (f(x)-g(x))$. Perhatikan bahwa, $f(x) - g(x) = -(g(x) - f(x))$, akibatnya $p(x)|g(x) - f(x)$ dan $g(x)\equiv f(x)~mod~(p(x))$. Jadi, sifat simetri terpenuhi.
	\item $f(x)\equiv g(x)~mod~(p(x))$ dan $g(x)\equiv h(x)~mod~(p(x))$, artinya $p(x)|f(x) - g(x)$ dan $p(x)|g(x) - h(x)$. Perhatikan bahwa, 
		$$\begin{array}{rcl}
		f(x) - h(x) &=& f(x) -( g(x) - g(x)) - h(x)\\
			      &=&  \underbrace{(f(x) - g(x))}_{p(x)|f(x) - g(x)} + \underbrace{(g(x)-h(x))}_{p(x)|g(x) - h(x)}.
		\end{array}$$
	Jadi, dapat kita simpulkan bahwa $p(x)| (f(x)-h(x))$, sehingga $f(x)\equiv h(x)~mod~(p(x))$ dan sifat transitif terpenuhi.
	\end{enumerate} $\blacksquare$
\\
\par 	Relasi "$\equiv$"  dari dua polinom ini, selanjutnya kita sebut sebagai kongruensi.
\\
\\
	\textbf{Teorema 5.2}
\par 	Misal $f(x),g(x),h(x),k(x),p(x) \in F[x]$, $f(x)\equiv g(x)~mod~(p(x))$ dan $h(x)\equiv k(x)~mod~(p(x))$. Maka berlaku:
	\begin{enumerate}
	\item $f(x)+h(x)\equiv g(x)+k(x)~mod~(p(x))$.
	\item $f(x)\cdot h(x)\equiv g(x) \cdot k(x)~mod~(p(x))$.
	\end{enumerate}
	\textit{Bukti}
\par 	Telah diketahui bahwa $f(x)\equiv g(x)~mod~(p(x))$ dan $h(x)\equiv k(x)~mod~(p(x))$, artinya $p(x)|f(x)-g(x)$ dan $p(x)|h(x)-k(x).$ Sehingga, $f(x)-g(x)= p(x)c(x)$ dan $h(x)-k(x)= p(x)d(x),$ untuk suatu $c(x),d(x) \in F[x].$
\par 	Akibatnya, diperoleh :
	\begin{enumerate}
	\item Pada operasi penjumlahan, berlaku\\
		$\begin{array}{rcl}
		f(x)+h(x) &=& (p(x)c(x)+g(x)) + (p(x)d(x)+k(x)\\
		&=& (p(x)c(x)+p(x)d(x))+(g(x)+k(x))\\
		&=& p(x)(c(x)+d(x)) + (g(x)+k(x)).
		\end{array}$
\\	Sehingga, $p(x)|(f(x)+h(x))-(g(x)+k(x))$. Sama artinya dengan $f(x)+h(x)\equiv g(x)+k(x)~mod~(p(x))$.
	\item Pada operasi perkalian,berlaku\\
	$\begin{array}{rcl}
		f(x)h(x) &=& (p(x)c(x)+g(x))\cdot (p(x)d(x)+k(x)\\
		&=& (p(x)c(x)p(x)d(x))+(p(x)c(x)k(x))+g(x)p(x)d(x)+(g(x)k(x))\\
		&=& p(x)(c(x)p(x)d(x)+c(x)k(x)+g(x)d(x)) + (g(x)k(x))\\
		&=& p(x)m(x) +(g(x)k(x)).
		\end{array}$
\\	Sehingga, $p(x)|(f(x)h(x))-(g(x)k(x))$. Sama artinya dengan \\$f(x)h(x)\equiv g(x)k(x)~mod~(p(x))$. $\blacksquare$
	\end{enumerate}	
\par 	Didefinisikan suatu kelas kongruen dari $f(x)$ adalah himpunan semua $g(x) \in F[x]$ yang memenuhi relasi $f(x) \equiv g(x)$ mod $(p(x))$, dapat juga ditulis $$[f(x)] := \{ g(x) \in F[x]~:~f(x)\equiv g(x) ~mod (p(x)) \} $$.
\par 	\textbf{Contoh :}
	\begin{itemize}
	\item Pada $\mathbb{R}[x]$, ambil $p(x) = x^2 + 1$. Pilih $f(x) = 2x+1$, maka $$[2x+1] := \{ k(x) \in F[x]~:~x^2+1 ~|~2x+1- k(x) \}.$$
	Perhatikan bahwa, 
	$$\begin{array}{lcl}
	(2x+1) - k(x) &=& (x^2+1) l(x)\\
	k(x) &=& (2x + 1) - (x^2+1) l(x),\\
	\end{array}$$
 	dari persamaan di atas, dapat kita simpulkan bahwa karakteristik polinom $k(x)$ adalah polinom yang apabila dibagi $x^2+1$ akan bersisa $2x+1$, sehingga $k(x) \in [2x+1].$
	\end{itemize}
	\textbf{Teorema 5.3}
\par 	Misal $F$ adalah lapangan, $f(x),g(x),p(x) \in F[x]$, maka pernyataan berikut berlaku :
	\begin{enumerate}
	\item $f(x) \equiv g(x)~mod (p(x)) \iff [f(x)]=[g(x)]$
	\item Untuk setiap $[f(x)],[g(x)] \in F[x]$, hanya ada dua kemungkinan,yaitu :
	\begin{enumerate}
		\item $[f(x)] = [g(x)]$
		\item $[f(x)] \cap [g(x)] = \emptyset. $
	\end{enumerate}
	\end{enumerate}
\par	\textbf{Contoh:}
	\begin{itemize}
	\item Pada $\mathbb{Z}_2[x]$, ambil $p(x) = x^2+x+1.$ Perhatikan bahwa, $$[x^2] = \{f(x) \in \mathbb{Z}_2[x]~:~ x^2 - f(x) = (x^2+x+1)\cdot g(x)\}.$$
	Ingat bahwa di $\mathbb{Z}_2[x]$ berlaku, $1+1 = 0$ dan $1=-1$, akibatnya $$x^2+1x+1 = x^2 -1x-1 = \underbrace {x^2 - (x+1)}_{x^2~\equiv x+1~mod(p(x))}.$$
	\end{itemize}
\par 	Dapat didefinisikan suatu himpunan dari kelas-kelas kongruen di $F[x]$ terhadap $p(x) \ne 0$, yaitu $$F[x]/p(x) := \{ [f(x)] ~:~f(x) \in F[x]~dan~f(x) \equiv g(x) ~mod~(p(x)), \exists g(x) \in F[x]\}.$$
\par 	Pada $F[x]/p(x)$, didefinisikan operasi jumlah dan kali sebagaimana dijelaskan di bawah.
\\ 	Untuk setiap $f(x),g(x) \in F[x]/p(x)]$, berlaku:
	\begin{enumerate}
	\item $[f(x)]+[g(x)] = [f(x)+g(x)]$
	\item $[f(x)]\cdot[g(x)] = [f(x)\cdot g(x)]$\\
	\end{enumerate}
	\textbf{Teorema 5.4}
\par 	Misal $p(x) \in F[x]$, dengan deg$(p(x)) >0.$
	\begin{enumerate}
	\item $F[x]/p(x)$ adalah ring komutatif dengan unsur kesatuan,$[1]$.
	\item Didefinisikan $F^* := \{ [a]~:a\in F \}$. $F^* \subset F[x]/p(x)$ dan $F\cong F^*.$\\
	\end{enumerate}
	\textit{Bukti:}
	\begin{enumerate}
	\item Ingat bahwa berdasarkan Teorema 1.3 $F[x]$ adalah daerah integral, sehingga, untuk setiap $[f(x)],[g(x)],[h(x)] \in F[x]/p(x)$  berlaku :
		\begin{itemize}
		\item $[f(x)]+[g(x)] = [f(x)+g(x)] \in F[x]/p(x)$. Berdasarkan Teorema 5.2.1. 
		\item Pada operasi penjumlahan, ($+$), berlaku sifat asosiatif,\\
		 	$\begin{array}{rcl}
			([f(x)]+[g(x)])+[h(x)] &=& [f(x)+g(x)]+[h(x)]\\
			&=& [(f(x)+g(x))+h(x)]\\
			&=& [f(x)+(g(x)+h(x))]\\
			&=& [f(x)]+[g(x)+h(x)]\\
			&=& [f(x)]+([g(x)]+[h(x)]).\\
			\end{array}$
		\item Terdapat $[0] \in F[x]/p(x)$, sedemikian sehingga untuk setiap $[h(x)] \in F[x]/p(x)$ berlaku $$[0]+[h(x)] = [0+h(x)]=[h(x)]=[h(x)+0]=[h(x)]+[0].$$
		\item Untuk setiap $[h(x)] \in F[x]/p(x)$, terdapat $[-h(x)] \in F[x]/p(x)$, sedemikian sehingga berlaku $$[h(x)]+[-h(x)] = [h(x)-h(x)]=[0]=[-h(x)+h(x)]=[-h(x)]+[h(x)].$$
		\item Berlaku sifat komutatif penjumlahan, \\$[f(x)]+[g(x)] = [f(x)+g(x)]=[g(x)+f(x)]=[g(x)]+[f(x)].$
		\end{itemize}
	Sampai di sini, telah ditunjukan bahwa $F[x]/p(x)$ adalah grup abelian (komutatif) penjumlahan. Selanjutnya
		\begin{itemize}
		\item Pada operasi perkalian, ($\cdot$), berlaku sifat asosiatif,\\
		 	$\begin{array}{rcl}
			([f(x)]\cdot [g(x)])\cdot [h(x)] &=& [f(x)\cdot g(x)]\cdot [h(x)]\\
			&=& [(f(x)\cdot g(x))\cdot h(x)]\\
			&=& [f(x)\cdot (g(x)\cdot h(x))]\\
			&=& [f(x)]\cdot [g(x)\cdot h(x)]\\
			&=& [f(x)]\cdot ([g(x)]\cdot[h(x)]).\\
			\end{array}$
		\item Pada operasi ($+$) dan ($\cdot$), berlaku sifat distributif kanan,\\
			$\begin{array}{rcl}
			([f(x)]+[g(x)])\cdot [h(x)] &=& [f(x)+ g(x)]\cdot [h(x)]\\
			&=& [(f(x)+ g(x))\cdot h(x)]\\
			&=& [(f(x)\cdot (h(x))+(g(x)\cdot h(x))]\\
			&=& [f(x)\cdot h(x)] + [g(x)\cdot h(x)]\\
			&=& [f(x)]\cdot [h(x)] + [g(x)] \cdot[h(x)],\\
			\end{array}$
			\\dan distributif kiri,\\
			$\begin{array}{rcl}
			[h(x)]\cdot ([f(x)]+[g(x)]) &=&  [h(x)] \cdot [f(x)+ g(x)]\\
			&=& [h(x) \cdot (f(x)+ g(x))]\\
			&=& [(h(x)\cdot (f(x))+(h(x)\cdot g(x))]\\
			&=& [h(x)\cdot f(x)] + [h(x)\cdot g(x)]\\
			&=& [h(x)]\cdot [f(x)] + [h(x)] \cdot[g(x)].\\
			\end{array}$
		\item Berlaku sifat komutatif perkalian, \\$[f(x)]\cdot [g(x)] = [f(x)\cdot g(x)]=[g(x)\cdot f(x)]=[g(x)]\cdot [f(x)].$\\
		\end{itemize}	
	Sampai di sini, telah ditunjukan bahwa $F[x]/p(x)$ adalah ring komutatif. 
		\begin{itemize}
		\item Terdapat $[1] \in F[x]/p(x)$, sedemikian sehingga untuk setiap $[h(x)] \in F[x]/p(x)$ berlaku $$[1]\cdot [h(x)] = [1 \cdot h(x)]=[h(x)]=[h(x)\cdot 1]=[h(x)]\cdot [1].$$
		\end{itemize}
	Jadi, telah ditunjukkan bahwa $F[x]/p(x)$ adalah ring komutatif dengan unsur kesatuan $[1]$.
	\item Berdasarkan definisi,  $F^* := \{ [a]~:a\in F \}$, karena $F$ adalah lapangan, maka $F^*$ bersifat tertutup terhadap operasi penjumlahan dan perkalian pada $F[x]/p(x)$, selain itu, $\exists [0_F] \in F^*$ yang merupakan elemen identitas pada operasi  				penjumlahan , dan untuk setiap $[a] \in F^*$, berlaku $[a] \ne [0_F]$ ,sehingga terdapat $[-a] \in F^*$. Maka, $F^*$ adalah subring dari $ F[x]/p(x)$.\\
		Definisikan suatu pengaitan, $\varphi$, dari $F$ ke $F^*$, di mana $\varphi : a \mapsto [a]$. Dapat ditunjukan bahwa $\varphi$ adalah suatu isomorfisma. 
		\begin{itemize}
		\item $\varphi$ pemetaan, karena $\forall a,b \in F$ dengan $a=b$ berlaku $$\varphi(a) = [a] = [b] =\varphi(b).$$
		\item $\varphi$ adalah suatu homomorfisma. Misal, ambil sembarang $\alpha a + \beta b \in F$ maka akan berlaku
		$$\begin{array}{rcl}
		\varphi (\alpha a + \beta b) &=& [\alpha a + \beta b]\\
		&=& [\alpha a] + [\beta b]\\
		&=& \alpha[a] +\beta [b]\\
		&=& \alpha \varphi(a) + \beta \varphi(b)
		\end{array}$$
		\item $\varphi$ satu-satu, karena $\forall [a],[b] \in F^*$, dengan $[a]=[b]$ berlaku\\ 
		$$a \equiv b~mod(p(x)), ~sehingga~p(x)|a-b.$$ Karena deg$(p(x)) >0$ dan deg$(a-b) =0$, maka haruslah $a-b = 0$. Jadi $a=b$.
		\item $\varphi$ pada, karena $\forall b\in F^*$ , berlaku $b=[y]$ untuk suatu $y\in F$, maka $\varphi(y)=[y]=b.$
		\end{itemize}
	Jadi, $F\cong F^*.$ $\blacksquare$
	\end{enumerate}
\par 	Sekarang akan didefinisikan istilah relatif prima pada $F[x]$. Untuk sembarang dua polinom $f(x),p(x) \in F[x]$, $f(x)$ dan $p(x)$ dikatakan relatif prima jika $fpb(f(x),p(x)) = 1$ atau terdapat $u(x),v(x) \in F[x]$ sedemikian hingga $f(x)u(x)+p(x)v(x) =1$.
\\
\\
	\textbf{Teorema 5.5}
\par 	Misalkan $f(x),p(x)\in F[x]$ relatif prima, dengan deg$(p(x)) >0$ dan $f(x)\ne 0$ , maka $[f(x)]$ unit di $F[x]/(p(x)).$ \\
	\textit{Bukti:}
\par 	Diketahui bahwa $f(x)$ dan $p(x)$ relatif prima, maka terdapat $u(x),v(x) \in F[x]$ sedemikian hingga $$f(x)u(x)+p(x)v(x) =1 \iff f(x)u(x) -1 = p(x)v(x),$$
	akibatnya, diperoleh bahwa $p(x)~|~f(x)u(x)-1$, sehingga
	$$\begin{array}{rcl}
	f(x)u(x)-1 &\equiv& 0 ~mod(p(x))\\
	f(x)u(x) &\equiv& 1~mod(p(x)).\\
	\end{array}$$
	Berdasarkan Teorema 5.3, berlaku
	$$\begin{array}{rcl}
	[f(x)][u(x)]&\equiv& [1] ~mod(p(x)).\\
	\end{array}$$
\par 	Maka, $[f(x)] \ne [0]$ adalah unit, karena $f(x)$ memiliki invers, yaitu $u(x).$ $\blacksquare$
\\
\par 	Pada Teorema 5.5, $[f(x)]$ yang dibahas adalah sembarang elemen di $F[x]/(p(x))$ yang tak nol. Sehingga, dapat kita simpulkan, bahwa setiap $[f(x)] \ne [0]$ di $F[x]/(p(x))$ adalah unit, maka $F[x]/(p(x))$ adalah lapangan.
\\
\newpage
	\textbf{Teorema 5.6}
\par 	Misalkan deg$(p(x))>0$, $F[x]/(p(x))$ adalah lapangan $\iff p(x)$ tak tereduksi.
\\
	\textit{Bukti:}
\par 	$(\Leftarrow)$ Diketahui $p(x)$ tak tereduksi. Ambil sembarang $[f(x)] \in F[x]/(p(x))$, di mana $[f(x)] \ne [p(x)]$, maka $ fpb(f(x),p(x))$ yang mungkin hanyalah 1 atau associate $p(x).$
\par 	Perhatikan bahwa, $fpb(f(x),p(x))|f(x)$ namun $p(x)$ tidak habis membagi $f(x)$, maka fpb yang mungkin adalah 1. 
\par 	Jadi, berdasarkan Teorema 5.5, [f(x)] adalah unit. Karena $[f(x)]$ sembarang elemen, maka $F[x]/(p(x))$ adalah lapangan.
\\
\par 	$(\Rightarrow)$  Misal $p(x)=f(x)g(x) - 0$, artinya $f(x)g(x) \equiv 0~mod(p(x)).$ \\
	Berdasarkan Teorema 5.3.1, $[f(x)g(x)]=[0]$,karena $F[x]/(p(x))$ adalah lapangan,  maka $[f(x)] = [0] \vee [g(x)] = [0].$
\par 	Pilih $[f(x)] = [0]$, maka
	$$\begin{array}{rcl}
	[f(x)] = [0] &\iff& f(x) \equiv 0~mod(p(x)).\\
	\end{array}$$
	Akibatnya,
	$$\begin{array}{rcl}
	p(x)|f(x)
	&\iff& f(x) = p(x)q(x).
	\end{array}$$
\par 	Di awal, telah kita misalkan $p(x)=f(x)g(x)$, substitusikan nilai dari $f(x)$, sehingga diperoleh bahwa $p(x) = (p(x)q(x))g(x)$. Maka dari itu, $q(x)g(x) = 1$ dan $g(x)$ adalah unit. 
\par 	Maka pembagi dari $p(x)$ hanyalah associate nya dan unit. Jadi, $p(x)$ tak tereduksi.$\blacksquare$
\\
\\
	\textbf{Catatan!}
	\begin{itemize}
	\item Misal $F$ adalah lapangan. Jika $F \subset E$, dan $E$ lapangan, maka $E$ adalah lapangan perluasan dari $F$. Contoh $\mathbb{Q} \subset \mathbb{R} \subset \mathbb{C}$.
	\item Berdasarkan Teorema 5.4 dan 5.6,  $F[x]/(p(x))$ adalah lapangan perluasan dari $F$.
	\end{itemize}
	\textbf{Teorema Kronecker}
\par 	Jika $f(x) \in F[x]$, deg$(f(x)) > 0$ , maka akan terdapat lapangan perluasan $E$ dari $F$, dan terdapat $\alpha \in E$ sedemikian sehingga $f(\alpha) =0.$
\\
	\textit{Bukti:}
\par 	Ingat kembali bahwa, $F[x]$ adalah Daerah Faktorisasi Tunggal, maka setiap $f(x)\in F[x]$ dapat dinyatakan sebagai hasil kali dari polinom polinom tak tereduksi di $F[x]$. $$f(x) = p(x)\cdot p_1(x)\cdot ... \cdot p_n(x).$$ 
\par 	Pilih $p(x)$ suatu polinom tak tereduksi, sehingga $F[x]/(p(x))$ adalah lapangan. (Berdasarkan Teorema 5.6).
\par 	Selanjutnya, definisikan suatu pengaitan, $\Psi$, dari $F$ ke $F[x]/(p(x))$, di mana $\Psi : a \mapsto a+(p(x)).$ Dapat ditunjukan bahwa $\Psi$ adalah suatu isomorfisma. 
		\begin{itemize}
		\item $\Psi$ pemetaan, karena $\forall~ a,b \in F$ dengan $a=b$, maka $a-b=0$, sehingga $p(x)|a-b$. Akibatnya,  berlaku $$\Psi(a) = a + (p(x)) = b + (p(x)) =\Psi(b).$$
		\item $\Psi$ adalah suatu homomorfisma. Misal, ambil sembarang $\alpha a + \beta b \in F$ maka akan berlaku
		$$\begin{array}{rcl}
		\Psi (\alpha a + \beta b) &=& (\alpha a + \beta b) + (p(x))\\
		&=& (\alpha a +(p(x))) + (\beta b +(p(x)))\\
		&=& \alpha(a+(p(x))) +\beta (b+(p(x)))\\
		&=& \alpha \Psi(a) + \beta \Psi(b)
		\end{array}$$
		\item $\Psi$ satu-satu, karena $\forall a+(p(x)),b+(p(x)) \in F[x]/(p(x))$, dengan $a+(p(x)) = b+(p(x))$ berlaku\\ 
		$$a-b \equiv 0~mod(p(x)), ~sehingga~p(x)|a-b.$$ Karena deg$(p(x)) >0$ dan deg$(a-b) =0$, maka haruslah $a-b = 0$. Jadi $a=b$.
		\item $\Psi$ pada, karena $\forall y_0\in  F[x]/(p(x))$ , berlaku $y_0=x_0+(p(x))$ untuk suatu $x_0\in F$, maka $\Psi(x_0)=x_0+(p(x))=y_0.$
		\end{itemize}
	Jadi, $F\cong F[x]/(p(x)).$
\par 	Maka dari itu, kita peroleh, $$F:=\{a:~a\in F\} = \{a+(p(x)): a \in F; p(x) \in F[x], dan~~p(x)~~tak~~tereduksi\}.$$
	Pilih $E=F[x]/(p(x))$. Misal ada $\alpha \in E$ yang tetap, dengan $\alpha = y+(p(x))$ ($y$ adalah representatif dari $\alpha$).
\par 	Sekarang, definisikan suatu pengaitan,$\phi_{\alpha}$, dari $F(x)$ ke $E$, di mana $$\phi_{\alpha}: \sum^{n}_{i=0} \ a_ix^i \mapsto \sum^{n}_{i=0} \ a_i\alpha^i,$$
	dengan $\alpha_i \in F$. Maka, 
	$$\begin{array}{rcl}
	\phi_{\alpha}( \sum^{n}_{i=0} \ a_ix^i) &=& \sum^{n}_{i=0} \ a_i\alpha^i\\
	&=& \sum^{n}_{i=0} \ a_i(y+(p(x)))^i\\
	&=& a_0+a_1(y+(p(x))) + a_2(y+(p(x)))^2+ \sum^{n}_{i=3} \ a_i(y+(p(x)))^i\\ 
	&=& (\sum^{n}_{i=0} \ a_iy^i) + (p(x)).
	\end{array}$$
\par 	Jika $p(x) = sum^{m}_{i=0} \ b_ix^i$, maka
	$$\begin{array}{rcl}
	\phi_{\alpha}(p(\alpha)) &=& \sum^{m}_{i=0} \ b_i{\alpha}^i\\
	&=& \sum^{n}_{i=0} \ b_i(y+(p(x)))^i\\
	&=& (\sum^{m}_{i=0} \ b_iy^i) + (p(x))\\
	&=& p(y)+p(x) =(p(x)) \equiv 0_E.
	\end{array}$$
	Jadi, $p(\alpha )=0$, dengan $\alpha \in E.$ $\blacksquare$
\\
\par 	\textbf{Contoh:}
	\begin{itemize}
	\item Ambil $(x^2+1) \in \mathbb{R}[x]$, yang tidak tereduksi. Berdasarkan Teorema 5.6., $ \mathbb{R}[x]/(x^2+1)$ adalah lapangan.$\mathbb{R} \longrightarrow \mathbb{R}[x]/(x^2+1)  r \mapsto r+<x^2+1>$
\par 	Misal $\alpha = x +<x^2+1>,$, perhatikan bahwa,
	$$\begin{array}{rcl}
	\alpha^2+1 &=& (x +<x^2+1>)^2 + 1\\
	&=& (x +<x^2+1>)^2 + (1 +<x^2+1>)\\
	&=& x^2 +2x<x^2+1>+<x^2+1><x^2+1>+ (1 +<x^2+1>)\\
	&=& x^2 +<x^2+1>+<x^2+1>+ (1 +<x^2+1>)\\
	&=& (x^2+1) + <x^2+1>\\
	&=& <x^2+1>\\
	&=& 0
	\end{array}$$
\par 	Perhatikan bahwa, $\mathbb{R} \subset \mathbb{R}[x]/(x^2+1)$. $\mathbb{R}[x]/(x^2+1)$ adalah lapangan perluasan dari $\mathbb{R}$, yang memuat $\alpha = i$. Dan dapat dibuktikan bahwa, $\mathbb{R}[x]/(x^2+1) \cong \mathbb{C}.$
	\end{itemize}


\chapter{Bilangan Aljabar dan Bilangan Transendental}
Diberikan $F$ lapangan dan $K$ lapangan perluasan dari $F$.
\begin{itemize}
\item  Derajat dari $K$ atas $F$, dinotasikan oleh $[K:F]$ adalah dimensi dari $K$ (sebagai ruang vektor) atas $F$.
\item $[K:F] < \infty \iff K$ perluasan berhingga dari $F$.
\end{itemize}

	\textbf {Teorema 6.1} 
\par	Jika $[L:K] < \infty $ dan $[F:K] < \infty  $ maka, $[L:F] = [L:K][K:F] < \infty $.
\\
\\ 	\textit{Bukti:}
\par 	Misal $[L:K]=m, [K:F]=n$, dan $\{e_1,e_2,...,e_m\}$ dan $\{f_1,f_2,...,f_n\}$ Berturut-turut basis untuk $L$ dan $K$. Ambil sembarang $V \in L$, maka $ \exists k_1,k_2,...,k_m \in K$ sedemikian sehingga $$V=k_1e_1+k_2e_2+...+k_me_m.$$ $\forall i \in \{1,2,...,m\},  		~\exists c_ij \in F~,j \in \{1,2,...,n\}$ sedemikian sehingga $$k_i=c_{i1}f_1+c_{i2}f_2+...+c_{in}f_n.$$
\par 	 Perhatikan bahwa 
	$$\begin{array}{rcl}
	V&=&k_1e_1+k_2e_2+...+k_me_m\\
 	&=&(c_{i1}f_1+c_{i2}f_2+...+c_{in}f_n)e_1+(c_{i1}f_1+c_{i2}f_2+...+c_{in}f_n)e_2+...\\
	&&+(c_{i1}f_1+ c_{i2}f_2+...+c_{in}f_n)e_m..(**)\\
	&=&c_{i1}(f_1e_1)+...+c_{in}f_ne_1+...+c_{m1}(f_1e_m)+...+c_mn(f_ne_m)..(*)
	\end{array}$$
\\ Sehingga, $span(\{e_1,e_2,...,e_m\} \times \{f_1,f_2,...,f_n\})=L.$ Selanjutnya, akan ditunjukan $(\{e_1,e_2,...,e_m\} \times \{f_1,f_2,...,f_n\})$ bebas linear.
\par 	Misal $(*)=0 \iff (**)=0$. Maka $c_{i1}f_1+c_{i2}f_2+...+c_{in}f_n = 0,\\~1\le i \le m..(***)$.
\par 	Karena $\{e_1,e_2,...,e_m\}$ bebas linear, diperoleh $(***),~c_ij=0~\forall~1 \le i \le m, 1 \le j \le n.$
\par 	 Kerena $\{f_1,f_2,...,f_n\}$ bebas linear, maka $\{e_1,e_2,...,e_m\} \times \{f_1,f_2,...,f_n\}$: Basis untuk L.
	$$\begin{array}{rcl}
	[L:F]&=&0(\{e_1f_1,...,e_1f_n,e_2f_1,...,e_2f_n,...,e_mf_1,...e_mf_n\})
	\\ &=&m \times n
	\\ &=&[L:K][K:F]
	\end{array}$$
$\blacksquare$
\\
\\
\textbf{Definisi 6.1}
\par 	 $u \in K$ disebut aljabar atas $F$ jika $\exists c_0,c_1,...,c_n \in F$ yang tidak semuanya nol sedemikian sehingga $c_0+c_1u+c_2u^2+...+c_nu^n=0_F$. Atau dengan kata lain $\exists f(x) \in F[x]$ sedemikian sehingga $f(u)=0_F$.
\\
\\
\textbf{Definisi 6.2}
\par 	$u \in K$ disebut transedental atas $F$, jika $u$ tidak aljabar atas $F$. Atau dengan kata lain $\forall f(x) \in F[x]$, maka $f(u) \ne 0_f$
\\
\\ Contoh:
\begin{enumerate}
\item $\sqrt{2}$ aljabar atas $\mathbb{Q}$ karena $\sqrt{2}$ pembuat nol untuk $x^2-2 \in \mathbb{Q}[x]$.
\item $i$ aljabar atas $\mathbb{R}$ (juga $\mathbb{Q}$) karena $i$ pembuat nol untuk $x^2+1 \in \mathbb{R}[x],\mathbb{Q}[x]$.
\item $\forall c \in F$ aljabar atas $F$, sebab $c$ pembuat nol untuk $x-c \in F[x]$
\item Apakah $\sqrt{1+\sqrt{3}}$ aljabar atas $\mathbb{Q}$?
$$\begin{array}{rcl}
 x=\sqrt{1+\sqrt{3}} &\Rightarrow& x^2=1+\sqrt{3}
\\ &\Rightarrow&x^2-1=\sqrt{3}
\\ &\Rightarrow&(x^2-1)^2=3
\\ &\Rightarrow&x^4-2x+1=3
\\ &\Rightarrow&x^4-2x^2-2=0
\end{array}$$
Jadi, $\sqrt{1+\sqrt{3}}$ pembuat nol dari $x^4-2x^2-2 \in \mathbb{Q}[x]$ 
\item $e,\pi$: transendental atas $\mathbb{Q}$ tapi aljabar atas $\mathbb{R}$.
\end{enumerate}
\textbf{Teorema 6.2}
\par Misal $u \in K.~u$ aljabar atas $F \Rightarrow $ terdapat $p(x) \in F[x]$ yang tak tereduksi sedemikian sehingga  $p(u)=0$.
\par $p(x)$ 'tunggal' hingga faktor konstan dan berderajat terkecil di $F[x]$ sedemikian sehingga $u$ pembuat nol-nya.
\\ \textit{Bukti:}
\par Definisikan 
	$\begin{array}{rcl}
	\\ \phi_u&:&F[x] \to K
	\\ &:& \sum^{n}_{i=0} \ a_ix_i \rightarrow \sum^{n}_{i=0} \ a_iu_i
	\end{array}$
\\
\\
Dapat ditunjukan bahwa $\phi_u$ well-defined.
\\ Selanjutnya, $ker(\phi_u) =\{f(x) \in F[x]|f(u)=0\}$ dengan $f(x) \in ker(\phi_u)$.
\begin{itemize}
\item $\forall g(x) \in F[x]$, maka
\\ $gf(u)=g(u)f(u)=g(u)0_F=0_Fg(u)=f(u)g(u)=(fg)(u)$.
\\ Artinya $ker(\phi u)$ ideal di $F[x]$.
\\ Berdasarkan Teorema 1.4. maka $ \exists p(x) \in F[x]$ sedemikian sehingga $ker(\phi u)=(p(x))$.
\\ Jelas $p(u)=0_F$. Karena $ker(\phi u)$ (definisi) begitu, maka generatornya pun begitu.
\item $p(x)$ berderajat terkecil sedemikian sehingga $u$ pembuat nol-nya.
\\ Bagaimanakah bentuk umum polinomial di $F[x]$ yang mana $u$ pebuat nol-nya tapi derajatnya $=deg(p(x))$?
\item Tunjukan $p(x)$ tak tereduksi.
\\ Andaikan $p(x)=r(x)s(x);~deg(r(x)),deg(s(x))>0$ tapi derajatnya kurang dari $p(x)$
\\ $0_F=p(u)=r(u)s(u)$
\\ Karena $F$ lapangan, maka $r(u)=0_F$ atau $s(u)=0_F$.
\\ Kontradiksi, karena, $p(x)$ derajat terkecil sedemikian sehingga $p(u)=0_F$.
\end{itemize}
Jadi, haruslah $p(x)$ tak tereduksi.$\blacksquare$
\\
\\
\textbf{Catatan}
\\
Polinom monik $p(x)$ dalam Teorema 6.2 disebut polinom minimal.
\\
\\ Contoh:
\begin{itemize}
\item $x^2-3 \in \mathbb{Q}[x]$ tak tereduksi, monik, $\sqrt{3}$ pembuat nol-nya. Artinya $x^2-3$ polinom minimal dari $\sqrt{3}$ atas $\mathbb{R}$. 
\item $x^2-3 \in \mathbb{R}[x]$
\\ $x^2-3=(x+\sqrt{3})(x-\sqrt{3}) \in \mathbb{R}[x]$. Artinya $x-\sqrt{3}$ polinom minimal dari $\sqrt{3}$ atas $\mathbb{R}$.
\end{itemize}

\textbf{Teorema 6.3}
\par $u \in K;~u$ aljabar atas $F;~p(x)$ polinom minimal dari $u;~deg(p(x))=n$, maka:
\begin{enumerate}
\item Misal $K_i$ sublapangan dari $K$, sedemikian sehingga $\forall i~F \subseteq K_i$ dan $u \in K_i$
\\ $remark:~K_1~\subsetneq K_2.~Artinya~K_i~tidak~harus~memuat~K_j~dan~sebaliknya.$
\\ Maka $F(u):= \bigcap^{}_{i} K_i \cong F[x]/(p(x))$.
\item $\{1_F,u,u^2,...,u^{n-1}\}$ basis untuk $F(u)$ atas $F$.
\item$[F(u):F]=n$ sama dengan derajat dari polinom minimalnya.
\end{enumerate}
\textbf{Catatan}
\par 	Untuk suatu aljabar $u$ dapat kita pilih polinom minimal sehingga dapat diperoleh lapangan baru atas $F$.
\\
\\
\textit{Bukti:}
\begin{enumerate}
\item $u^n \in F(u)~ \forall n \in \mathbb{N}$.
\par Definisikan
$\begin{array}{rcl}
\\ \phi&:&F[x] \to F(u)
\\ &:& f(x) \to f(u)
\end{array}$
\\
\\Jelas bahwa $\phi$ well-defined dan homomorfisma.
\\ $ker( \phi ):=\{f(x) \in F[x]| \phi (f(x))=0_F\}=p(x).$
\\ $ker ( \phi )$ ideal di $F[x]$.
\item \par Definisikan
$$\begin{array}{rcl}
\\ \psi&:&F[x]/ker( \phi ) \to im( \phi ) \subseteq F(u)
\\ \psi&:& F[x]/p(x) \to im( \phi )
\\ &:&[f(x)] \to f(u)
\end{array}$$
\\
apakah $\psi$ onto? 
\\ $\forall y \in Im( \phi )$ maka $\exists f(x) \in F[x]$ sedemikian sehingga $y=f(u)$.
\\ Ini berarti $\psi ([f(x)])=f(u)=y$
\\
\\ Berdasarkan Teorema isomorfisma $I$ untuk ring. Yaitu $R/K \to S$ onto dan $K$ ideal dari $R$ maka, $R/K \cong S$.
\\Jadi, $F[x]/(p(x)) \cong Im( \phi )$
\\
$$\begin{array}{rcl}
\\ \phi&:&F[x] \to F(u)
\\ &:& \alpha \to \alpha
\\ &:& x \to u
\end{array}$$ di mana $\alpha$ dan $u \in Im( \phi ) \subseteq F(u)$
\\
\\ $im( \phi ):$ sublapangan dari $F(u)$ sedemikian sehingga $F \subseteq Im( \phi )$.
\\
\\ $F(u)= \bigcap^{}_{i}K_i \Rightarrow F(u)$ sublapangan terkecil yang memuat $F$ dan $u$, maka $F(u) \subseteq Im( \phi )$
\\ Artinya $F(u)=Im( \phi )$.
\end{enumerate} $\blacksquare$



\chapter{Perluasan Aljabar}
	\textbf{Definisi 7.1} 	
\par 	Misal $K$ adalah perluasan lapangan dari $F$. $K$ adalah perluasan aljabar dari $F$ jika dan hanya jika, $\forall~\alpha \in K$, $\alpha$ adalah aljabar atas $F$.
\\
\\
\textbf{Contoh}% cari dulu di foto kayaknya.
\\
\\
\textbf{Teorema 7.1}
\par 	Jika $[K:F] < \infty$, maka $K$ adalah perluasan aljabar atas $F$.
\\
\textit{Bukti}
\par 	Karena $[K:F] < \infty$, maka terdapat basis $E= \{ e_1,...,e_n\}$ atas $F$.
\par 	Ambil sembarang $u \in K$, maka akan ada dua kasus, yaitu :
	\begin{enumerate}	
	\renewcommand{\labelenumi}
	{\roman{enumi}}
	\item Jika $u^i=u^j~,~0 \le i<j$, maka $u$ adalah pembuat nol dari $x^i-x^j\in F[x].$
	\item Jika tidak memenuhi i, artinya tidak ada $i~\& ~j,$ dengan $0 \le i<j~$, sedemikian hingga $u^i=u^j$.
\\	Misalkan kita miliki suatu himpunan yang membangun $K$, namun tidak bebas linear, yaitu $\{1_F, u,u^1,u^2,...,u^n\}$. Mengapa tidak bebas linear? Karena banyaknya anggota himpunan tersebut adalah $n+1$, dan dijamin bahwa himpunan yang anggota nya lebih banyak 		dari $n$, di mana $n$ adalah dimensi dari $K$, akan tidak bebas linear (bergantung linear).	
\\	Akibatnya, akan ada $c_i\in F$, tidak semuanya nol, sedemikian sehingga, $$c_01_F + c_1u + c_2u^2+...+c_nu^n = 0.$$
	Jadi, $u$ adalah pembagi nol untuk polinom dengan bentuk, $$c_0+ c_1x + c_2x^2+...+c_nx^n \in F[x].$$	
	\end{enumerate}
\par 	Sehingga, untuk setiap $u\in K,~u~$adalah aljabar atas $F$. Jadi, $K$ adalah perluasan aljabar atas $F$. $\blacksquare$
\par 	Misal $F$ adalah lapangan, $u\in K$ dan $u \notin F$ aljabar atas $F$, maka $F(u):=\{a+bu: a,b\in F\}$ adalah lapangan terkecil yang memuat $F$ dan $u$. Berdasarkan Teorema 6.3, $[F(u):F] = n <\infty$, sehingga, berdasarkan Teorema 7.1, $F(u)$ adalah perluasan aljabar 			atas F.
\\
\par 	Sekarang, coba perhatikan apabila $K$ perluasan lapangan atas $F$, $u_1,...,u_n \in K$ dan $u_1,...,u_n \notin F$ merupakan aljabar atas $F$, maka $F[u_1,...,u_n]$ adalah lapangan terkecil yang memuat $F$ dan $u_1,...,u_n.$ Sehingga, $[F(u_1,...,u_n):F]<\infty$ 			dan  $F(u_1,...,u_n)$ adalah perluasan aljabar atas $F$.
\\
\par 	Bagaimana bisa diperoleh kesimpulan di atas?
\\
\par 	Jadi, untuk memperoleh kesimpulan di atas, dilakukan penyisipan $u_1,...,u_n$ secara bertahap ke lapangan F. Pertama, disisipkan dulu $u_1$ ke dalam $F$, di mana $u_1$ merupakan aljabar atas $F$, sehingga berdasarkan Teorema 6.3, $[F(u_1):F] = <\infty$. Lalu, disisipkan $u_2$ ke dalam $F(u_1)$, di mana $u_2$ merupakan aljabar atas $F(u_1)$, sehingga berdasarkan Teorema 6.3, $[F(u_1,u_2):F(u_1)] = <\infty$. Karena $[F(u_1):F] = <\infty$ dan $[F(u_1,u_2):F(u_1)] = <\infty$, maka,berdasarkan Teorema 6.1, $[F(u_1,u_2):F]<\infty$. Selanjutnya, lakukan penyisipan $u_3,...,u_n$ dengan cara yang sama seperti sebelumnya, dan pada akhirnya diperoleh kesimpulan bahwa $[F(u_1,...,u_n):F]<\infty$ dan  $F(u_1,...,u_n)$ adalah perluasan aljabar atas $F$.
\\
\par 	Perhatikan kembali Teorema 7.1, apakah konvers dari pernyataan tersebut juga benar? (Catatan: konvers dari Teorema 7.1 adalah Jika $K$ adalah perluasan aljabar atas $F$, maka $[K:F] < \infty$.)
\\
\par 	Untuk sembarang $n\ne1 \in \mathbb{N}, 2^{\frac{1}{n}} \notin \mathbb{Q}$. Namun, $2^{\frac{1}{n}}$ adalah pembuat nol untuk $x^n-2 \in \mathbb{Q}[x]$, sehingga $2^{\frac{1}{n}}$ aljabar atas $\mathbb{Q}.$
\\
\par 	Perhatikan bahwa, $x^n-2$ adalah polinom tak tereduksi,berdasarkan Kriteria Eisenstein dengan $p=2$. Selain itu,  $x^n-2$ adalah polinom monik. Akibatnya,  $x^n-2$ adalah polinom minimal untuk $2^{\frac{1}{n}}$. 
\\
\par 	Maka dari itu, berdasarkan Teorema 6.3,berlaku
	\begin{enumerate}
	\item $\mathbb{Q}/((x^n-2)) \cong \mathbb{Q}(2^{\frac{1}{n}}).$ 
	\item $\{1_{\mathbb{Q}},2^{\frac{1}{n}},(2^{\frac{1}{n}})^2,...,(2^{\frac{1}{n}})^{n-1}\}$ adalah basis dari  $\mathbb{Q}(2^{\frac{1}{n}})$ atas $\mathbb{Q}$.
	\item $[ \mathbb{Q}(2^{\frac{1}{n}})~:~\mathbb{Q}]=n<.$
	\end{enumerate}
\par 	Pilih $K=\mathbb{Q}(2^{\frac{1}{n}},2^{\frac{1}{n}},...)$, maka untuk setiap $n\in $
	$\mathbb{N} \setminus \emptyset $, berlaku $n=[ \mathbb{Q}(2^{\frac{1}{n}})~:~\mathbb{Q}] \le [K:\mathbb{Q}.$
	maka $[K:\mathbb{Q}] = \infty.$ 
\\
\par 	Dari bahasan di atas, konvers dari Teorema 7.1 tidak benar.
\\
\\
	\textbf{Teorema 7.2}
\par 	Jika $K=F(u_1,...,u_n)$ adalah perluasan lapangan (berhingga) dari $F$, $u_i$ aljabar atas $F$, $\forall i=1,2,...,n,$ maka $K$ adalah perluasan aljabar dari F (berhingga).
\\
	\textit{Bukti:}
\par 	Pada paragraf sebelum-sebelumnya, kita tahu bahwa $$F\subset F(u_1) \subset F(u_1,u_2)\subset...\subset F(u_1,...,u_n)=K, $$dan karena $u_i$ aljabar atas $F$, maka $u_i$ juga aljabar atas $F(u_1,...,u_n)$. Maka, berdasarkan Teorema 6.3, 
	$$[F(u_1,...,u_{i-1},u_i)~:~F(u_1,...,u_{i-1})] < \infty.$$
\par 	Berdasarkan Teorema 6.1,
	$$\begin{array}{rcl}
	[K:F] &=& [K:F(u_1,...,u_{i-1})][F(u_1,...,u_{i-1}): F(u_1,...,u_{i-2})]...[F(u_1,u_2):F(u_1)]\\
	&~&[F(u_1):F]
	< \infty.
	\end{array}$$
\par 	Jadi, berdasarkan Teorema 7.1, $K$ adalah perluasan aljabar dari $F$.$\blacksquare$
\\
	\textbf{Contoh:}
	\begin{itemize}
	\item $\sqrt{3},\sqrt{5}$ merupakan aljabar atas $\mathbb{Q}$, karena $\sqrt{3}$ merupakan pembuat nol untuk $x^2-3$  dan $\sqrt{5}$ merupakan pembuat nol untuk $x^2-5$, di mana  $x^2-3, x^2-5\in \mathbb{Q}[x]$. Maka, $[\mathbb{Q}(\sqrt{3},\sqrt{5}):			\mathbb{Q}]<\infty$.\\ \\
	Perhatikan bahwa,$\mathbb{Q}(\sqrt{3}) = a+b\sqrt{3}$, maka $\mathbb{Q}(\sqrt{3})=span\{1,\sqrt{3}\}$ dan $[\mathbb{Q}(\sqrt{3}):\mathbb{Q}]=2$. 
	\\ \\
	Bagaimana dengan $[\mathbb{Q}(\sqrt{3},\sqrt{5}):\mathbb{Q}]$? Harus dicari terlebih dahulu polinom minimal dari $\sqrt{5}$ atas $\mathbb{Q}(\sqrt{3})$. Di awal telah kita ketahui bahwa $\sqrt{5}$ merupakan pembuat nol untuk $x^2-5$, karena  $x^2-5$ monik dan tak tereduksi di $\mathbb{Q}(\sqrt{3})[x]$,berdasarkan Kriteria Eisenstein dengan $p=5$, maka $x^2-5$ adalah polinom minimal atas $\mathbb{Q}(\sqrt{3})$.\\ \\
	Andaikan $\pm\sqrt{5} \in \mathbb{Q}(\sqrt{3})$, maka  $\pm\sqrt{5} = a +b\sqrt{3} : a,b\in \mathbb{Q}~,~a,b \ne 0.$ Sehingga,
	$$\begin{array}{rcl}
	(\sqrt{5})^2 &=&  (a +b\sqrt{3})\\
	5&=& a^2+2ab\sqrt{3}+3b^2\\
	\sqrt{3}&=& \frac{5-a^2-3b^2}{2ab} \in\mathbb{Q}.
	\end{array}$$
	Padahal $\sqrt{3} \notin \mathbb{Q}.$ Maka haruslah $\pm\sqrt{5} \notin \mathbb{Q}(\sqrt{3})$.\\ \\
	Maka, berdasarkan Teorema 6.1, $[\mathbb{Q}(\sqrt{3},\sqrt{5}):\mathbb{Q}(\sqrt{3})]=$ deg(polinom minimal)=$2$. Lalu, berdasarkan Teorema 6.1, 
	$$\begin{array}{rcl}
	[\mathbb{Q}(\sqrt{3},\sqrt{5}):\mathbb{Q}] &=& [\mathbb{Q}(\sqrt{3},\sqrt{5}):\mathbb{Q}(\sqrt{3})]\times[\mathbb{Q}(\sqrt{3}):\mathbb{Q}]\\
	&=& 2\times 2\\
	&=& 4.
	\end{array}$$
	\end{itemize}

	\textbf{Teorema 7.3}
\par 	Misalkan $L$ adalah perluasan aljabar atas $K$ dan $K$ adalah perluasan aljabar atas $F$, maka $L$ adalah perluasan aljabar atas $F$.\\
	\textit{Bukti:}
\par 	Karena $L$ adalah perluasan aljabar atas $K$ akan terdapat $u$ suatu aljabar atas $K$, dengan $u\in L$, dan $L$ ruang vektor atas $K$, maka terdapat $a_i \in K$ ,tidak semuanya nol, sedemikian sehingga $a_0+a_1u+...+a_mu^m=0.$
\par 	Karena $K$ adalah perluasan aljabar atas $F$, dengan $K=F(a_1,...,a_m)$, maka $F \subset F(a_1,...,a_m)$ dan $a_1,...,a_m \in F(a_1,...,a_m)$.
\par 	Maka $u$ adalah aljabar atas $F(a_1,...,a_m)$. Berdasarkan Teorema 7.2, $$[F(a_1,...,a_m,u):F(a_1,...,a_m)] < \infty.$$
\par 	Selain itu, $a_i \in F(a_1,...,a_m)$ merupakan aljabar atas $F$ untuk setiap $i=1,2,...,m.$ Sehingga,berdasarkan Teorema 7.2,$[F(a_1,...,a_m) : F]< \infty$.
\par 	Akibatnya, berdasarkan Teorema 6.1, $$[F(a_1,...,a_m,u): F]< \infty.$$ 
	Karenanya,berdasarkan Teorema 7.1,$F(a_1,...,a_m,u)$ adalah perluasan aljabar atas F. Karena $u$ sembarang, maka $L \subseteq F(a_1,...,a_m,u).$ $\blacksquare$
\\
\\
	\textbf{Teorema 7.4}(Hermite)
\par 	$e$ transendental (atas $\mathbb{Q}).$
\\
	\textit{Bukti:}
\par 	Misal $f(x) \in \mathbb{R}$[x] : deg$(f(x))=r. Misalkan juga, F(x)=f(x) + {(1)}(x)+...+f^{(r)}(x).$Akan diperoleh,
	$$\begin{array}{rcl}	
	\frac{d}{dx}(e^{-x}F(x)) &=& -e^{-x}F(x) + e^{-x}(f^{(1)}(x)+...+f^{(r)}(x))\\
	&=&  -e^{-x}f(x).
	\end{array}$$
\par 	Diberikan $k \in \mathbb{N}$, maka $[0,k]$ diperoleh
	$$\begin{array}{rcl}
	\frac{e^{-k}F(k) - F(0)}{k} &=& -e^{-\theta_k k}f(\theta_k k)\\
	F(k) - e^{-k}F(0) &=& -e^{(1-\theta_k) k}f(\theta_k k) k.
	\end{array}$$
\par 	Untuk $k=1,2,...,n$ ,akan diperoleh,
	$$\begin{array}{rcl}
	k=1~~(F(1) - eF(0) &=&  (-1e^{1(1-\theta_1) }f(1\theta_1))~~=\varepsilon_1 \\
	k=2~~(F(2) - eF(0) &=&  (-2e^{2(1-\theta_2) }f(2\theta_2))~~=\varepsilon_2\\
	\vdots~~~~~~~~~~~~~~~~~~~~~~~~ &\vdots& ~~~~~~~~~~~~~~~~~~~~~~~~~~~~\vdots\\
	k=n~~(F(n) - eF(0) &=&  (-ne^{n(1-\theta_n) }f(n\theta_n))~~=\varepsilon_n.
	\end{array}$$
\par 	Misal $e$ adalah aljabar, maka $\exists c_0,c_1,...,c_n \in \mathbb{Z}~~(c_0>0)$ sedemikian sehingga,
	$$\begin{array}{rcl}
	c_0+c_1e+c_2e^2+...+c_ne^n &=&0\\
	c_1e+c_2e^2+...+c_ne^n&=& -c_0
	\end{array}$$
\par 	Selanjutnya dilakukan perkalian $c_k$ dengan $\varepsilon_k$ untuk k yang bersesuaian,
	$$\begin{array}{rcl}
	k=1~~c_1(F(1) - eF(0) &=&  c_1(-1e^{1(1-\theta_1) }f(1\theta_1))~~=c_1\varepsilon_1 \\
	k=2~~c_2(F(2) - eF(0) &=&  c_2(-2e^{2(1-\theta_2) }f(2\theta_2))~~=c_2\varepsilon_2\\
	\vdots~~~~~~~~~~~~~~~~~~~~~~~~~~ &\vdots& ~~~~~~~~~~~~~~~~~~~~~~~~~~~~~~\vdots\\
	k=n~~c_n(F(n) - eF(0) &=&  c_n(-ne^{n(1-\theta_n) }f(n\theta_n))~~=c_n\varepsilon_n.
	\end{array}$$
	Lalu, lakukan penjumlahan dan akan diperoleh,
	$$\begin{array}{rcl}
	&~~& c_1F(1) +c_2F(2) +...+c_nF(n) - F(0)(c_1e+c_2e^2+...+c_ne^n)\\
	&=& c_1\varepsilon_1+c_2\varepsilon_2+...+c_n\varepsilon_n.
	\end{array}$$
\par 	Dengan suatu proses perhitungan, pada akhirnya akan diperoleh
	$$f(x) = \frac{1}{p-1!}x^{p-1}(1-x)^p(2-x)^p...(n-x)^p.$$


\chapter{Splitting Field}
	Pada chapter ini, $K$ selalu kita misalkan sebagai perluasan lapangan dari F.
\\
\\	\textbf{Definisi 8.1}
\par 	Misal $f(x) \in F[x]$, deg$(f(x))=n>0$, dan $$f(x)=c(x-u_1)(x-u_2)...(x-u_n)~~\in K[x],$$ maka $f(x)$ adalah \textit{split} atas lapangan $K$.
\\
\par 	Artinya, $f(x)$ dikatakan \textit{split} atas lapangan $K$, jika $f(x)$ dapat dinyatakan atas faktor-faktor linier di $K[x]$.
\\
\\
	\textbf{Definisi 8.2}
\par 	Misal $f(x) \in F[x],$ $K$ merupakan lapangan \textit{splitting} (lapangan akar) dari $f(x)$ atas $F$, jika:
	\begin{enumerate}	
	\renewcommand{\labelenumi}
	{\roman{enumi}}
	\item $f(x)$ \textit{split} atas K.
	\item $K = F(u_1,...,u_n)$, artinya $K$ merupakan perluasan lapangan terkecil sedemikian sehingga $F\subseteq K$ dan $u_1,...,u_n \in K.$
	\end{enumerate}
\par 	Dari dua definisi di atas, tidak disebutkan secara spesifik apakah $f(x)$ itu merupakan polinom tak tereduksi ataupun tereduksi, mengapa? Sebelum menjawab pertanyaan ini, coba perhatikan contoh berikut.
\\
\\
	\textbf{Contoh :}
	\begin{enumerate}
	\item Perhatikan $x^2+1 \in \mathbb{R}[x]$, polinom ini merupakan polinom tak tereduksi di $\mathbb{R}[x]$. Karena $$x^2+1=(x+\textit{i})(x+(\textit{-i})) \in \mathbb{C}[x],$$ maka $K=\mathbb{R}(\textit{i}) = \mathbb{C}$ dan $\mathbb{C}$ merupakan 			lapangan \textit{splitting} dari $x^2+1$ atas $\mathbb{R}$.
	\item Untuk $x^2-2 \in \mathbb{Q}[x]$, polinom ini merupakan polinom tak tereduksi di $\mathbb{Q}[x]$. Karena $$x^2-2 =(x-\sqrt{2})(x+\sqrt{2})~~\in \mathbb{Q}(\sqrt{2}),$$ maka $K=\mathbb{Q}(\sqrt{2})$ dan $\mathbb{Q}(\sqrt{2})$ merupakan 					lapangan \textit{splitting} dari $x^2-2$ atas $\mathbb{Q}$.
	\item 	$K=\mathbb{Q}(\sqrt{2},\textit{i})$ merupakan lapangan \textit{splitting} dari $x^4-x^2-2$ atas $\mathbb{Q}$. Dikarenakan,
	$$\begin{array}{rcl}
	x^4-x^2-2 &=& (x^2+1)(x^2-2)\\
	&=& (x+\textit{i})(x+(\textit{-i}))(x-\sqrt{2})(x+\sqrt{2})~~\in \mathbb{Q}(\sqrt{2},\textit{i}).
	\end{array}$$
	\item $\mathbb{C}$ bukanlah lapangan \textit{splitting} dari $x^2+1$ atas $\mathbb{Q}$. Dengan cara yang sama dengan Contoh 1, dapat kita peroleh bahwa $K=\mathbb{Q}(\textit{i})$. Jelas bahwa $\mathbb{Q}(\textit{i}) \subseteq \mathbb{C}$. Namun $\mathbb{C}$ 	bukanlah subset dari $\mathbb{Q}(\textit{i})$, karena terdapat $\sqrt{2} \in \mathbb{C}$ tetapi $\sqrt{2} \notin \mathbb{Q}(\textit{i}).$ Sehingga $\mathbb{C} \ne \mathbb{Q}(\textit{i}).$
	\item Perhatikan polinom $cx+d \in F[x],$ dengan $c,d\in F.$  $cx+d$ merupakan split atas F, karena $$cx+d=c(x+c^{-1}d),$$ di mana $c^{-1}d \in F$. Akibatnya, $F$ adalah lapangan \textit{splitting} dari $cx+d$ atas $F$.
	\end{enumerate}
\par Dari contoh-contoh di atas, dapat disimpulkan beberapa hal berkaitan dengan ke-tak tereduksi-an dari $f(x)$. Jika $f(x)$ tak tereduksi  di $F[x]$, maka akan ada lapangan perluasan $K$ dari $F$ yang memuat akar dari $f(x)$ tersebut, seperti contoh 1 dan 2. Jika $f(x)$ 		tereduksi di $F[x]$, akan ada dua kemungkinan, pertama, ketika $f(x)$ tereduksi dan dapat dinyatakan sebagai faktor-faktor linier di $F[x]$, maka $F$ adalah lapangan \textit{splitting} dari dirinya sendiri, seperti contoh 5, kedua, ketika $f(x)$ tereduksi namun belum 		dapat dinyatakan sebagai faktor-faktor linier di $F[x]$, maka akan ada lapangan perluasan $K$ dari $F$ yang memuat akar dari $f(x)$ tersebut, seperti contoh 3.
\\
\\
	\textbf{Teorema 8.1}
\par	Misal $f(x) \in F[x]$ dan deg$(f(x))=n>0$, maka akan terdapat $K$ suatu lapangan \textit{splitting} dari $f(x)$ atas $F$ sedemikian sehingga $[K:F] \le n!.$
	\textit{Bukti}
\\ 	Untuk membuktikan teorema ini akan digunakan induksi.
	\begin{itemize}
	\item Tunjukan benar untuk n=1. deg$(f(x))=1$, maka $K=F$ adalah lapangan \textit{splitting} dari $f(x)$ atas $F$ dan $[F:F]=1\le 1!.$
	\item Asumsikan benar untuk deg$(f(x)) =n-1.$  Maka akan terdapat $K$ suatu lapangan \textit{splitting} dari $f(x)$ atas $F$,  sedemikian sehingga $[K:F]\le (n-1)!.$
	\item Tunjukan benar untuk deg$(f(x))=n$.\\
	Perhatikan bahwa $F$ adalah daerah faktorisasi tunggal, maka untuk $f(x)\in F[x]$ dapat dinyatakan sebagai, $$f(x)=p(x)\cdot (polinomial-polinomial~tak~tereduksi~lainnya),$$ dengan $p(x)$ merupakan polinom tak tereduksi.Tanpa mengurangi keumuman dapat 			diasumsikan $p(x)$ adalah polinom monik, sehingga $p(x)$ adalah polinom minimal untuk suatu u.\\ \\
	Karena $p(x)$ tak tereduksi, maka dapat dibentuk lapangan $F[x]/p(x)$ sedemikian sehingga $F\subset F[x]/p(x)$ dan $u\in F[x]/p(x)$ di mana $p(u)=0_F.$ \\ \\
	Berdasarkan Teorema 6.3, $F[x]/p(x)\cong F(u)$ dan 
	$$\begin{array}{rcl}
	[F[x]/p(x)~:~F] = [F(u)~:~F]&=&deg(p(x))\\ 
	&\le& deg(f(x))\\ 
	&=& n.
	\end{array}$$
	\\ \\
	Selanjutnya, dikarenakan $p(u)=0_F$, diperoleh, $$f(x)=(x-u)g(x),$$ untuk suatu $g(x)\in F(u)[x].$\\ \\
	Perhatikan bahwa deg$(g(x))=n-1$, artinya keberadaan lapangan \textit{splitting} $K$ dari $g(x)$ atas $F(u)$ dijamin oleh asumsi sebelumnya, sedemikian sehingga $[K:F(u)]\le (n-1)!$.Sehingga,$$ g(x)=c(x-u_1)(x-u_2)...(x-u_{n-1})\in K[x]$$ jika dan hanya jika 
	$$f(x)=c(x-u)(x-u_1)(x-u_2)...(x-u_{n-1})\in K'[x].$$\\
	Perhatikan juga, $K'=F(u_1,u_2,...,u_{n-1})=F(u,u_1,u_2,...,u_{n-1})$. Jadi, $K'$ adalah lapanga \textit{splitting} dari $f(x)$ atas $F$. Sehingga,
	$$\begin{array}{rcl}
	[K':F]&=&[K':K][K:F(u)][F(u):F]\\
	&\le& 1\cdot (n-1)!\cdot n\\
	&=& n!
	\end{array}$$
	\end{itemize}
	Jadi, terbukti.$\blacksquare$
\\
\\
	\textbf{Teorema 8.2}
\par	Misal $F,E$ adalah lapangan, $\sigma : F \longrightarrow E$ suatu isomorphisma, $f(x) \in F[x]$, deg$(f(x))>0$. $\sigma f(x) \in E[x]$.\\
	$K$ : lapangan \textit{splitting} dari $f(x)$ atas $F$.\\
	$L$ : lapangan \textit{splitting} dari $\sigma f(x)$ atas $F$.\\
	Maka, $K\cong L.$
\chapter{Ujian Tengah Semester}


\chapter{Kenormalan pada Perluasan Lapangan}
	Sebelum memulai \textit{chapter} 11, akan diberikan bukti dari Teorema 8.2, di mana pada \textit{chapter} 8 belum dibuktikan. Selanjutnya, Teorema 8.2 akan digunakan untuk membuktikan beberapa Teorema pada \textit{chapter} ini.
\\ \\	\textit{Bukti: }\textbf{(Teorema 8.2)}\\
	Diketahui bahwa $\sigma : F \longrightarrow E$ adalah suatu isomorfisma, maka 
	$$\begin{array}{rcl}
	\sigma &:& F[x] ~~~~~~~\longrightarrow~~ E[x]\\
	&:&\sum^{n}_{i=0} \ a_ix^i \mapsto \sum^{n}_{i=0} \ \sigma (a_i)x^i=\sigma f(x),
	\end{array}$$ juga suatu isomorphisma.
\\	Selanjutnya, akan digunakan induksi untuk menyelesaikan bukti Teorema ini.
	\begin{itemize}
	\item Untuk deg$(f(x))=1$ dan deg$(\sigma f(x))=1.$
\\	Dikarenakan deg$(f(x))=1$, maka $f(x)=c(x-u)$ dan $u$ merupakan akar dari $f(x)$. $u$ belum tentu elemen di $F$, namun jelas bahwa $u\in F(u)=K.$ Akibatnya, $f(x)\in K[x].$
\\	Perhatikan bahwa, $$f(x)=c(x-u)=cx-cu,$$
	karena $c\in F$ dan $cu\in F$, maka $u\in F$ dan $f(x) \in F[x].$ Sehingga, $K=F(u)=F.$
\\ \\	Selanjutnya, karena $F[x] \cong E[x]$,maka deg$(\sigma f(x))=1$ dan  
	$$\begin{array}{rcl}
	\sigma(f(x))&=&\sigma(c(x-u))\\
	&=&\sigma(cx-cu)\\
	&=&\sigma(cx)-\sigma(cu)\\
	&=&\sigma(c)x-\sigma(c)\sigma(u)\\
	&=&cx-c\sigma(u).
	\end{array}$$
	$\sigma(u)$ belum tentu elemen di $E$, namun jelas bahwa $\sigma(u)\in E(u)=L.$ Akibatnya, $\sigma f(x)\in L[x].$
	Tetapi,karena $c\in E$ dan $c\sigma(u)\in E$, maka $\sigma(u)\in E$ dan $\sigma f(x) \in E[x].$ Sehingga, $L=E(\sigma(u))=E.$
\\	Jadi, $L\cong E \cong F \cong K.$
	\item Asumsikan benar untuk deg$(f(x))=n-1.$
	\item Misal deg$(f(x))=n$.
	%Ada yang ga paham :(
\\	Diketahui, $F[x] \cong E[x]$, maka $\sigma(p(x))$ juga polinomial minimal di $E[x]$, dikarenakan koefisien dari derajat tertinggi $p(x)$ adalah 1 dan $\sigma$ suatu homomorphisma sehingga $\sigma(1)=1$.
\\	Perhatikan bahwa,
	$$f(x)=p(x)\cdot (polinomial-polinomial~tak~tereduksi~yang~lain).$$
	Setiap akar dari $p(x)$, juga merupakan akar dari $f(x)$. Sehingga, akar-akar dari $p(x)\in K.$
\\ \\	Sekarang coba perhatikan,
	$$\sigma f(x)=\sigma p(x)\cdot \sigma((polinomial-polinomial~tak~tereduksi~yang~lain)).$$
	Setiap akar dari $\sigma p(x)$, juga merupakan akar dari $\sigma f(x)$. Sehingga, akar-akar dari $\sigma p(x)\in L.$
\\ \\	Misal $u\in K$ sedemikian hingga $p(u)=0$ dan $v\in L$ sedemikian sehingga $\sigma p(v)=0$. Apakah pemetaan $F(u)\longrightarrow E(v)$ suatu isomorfisma? Ya, Berdasarkan Teorema 10.4
\\ \\	Selanjutnya, karena $p(u)=0$ maka $f(u)=0.$ Sehingga, $f(x)=(x-u)g(x)$ untuk suatu $g(x) \in F(u)[x].$ Diperoleh bahwa
	$$\begin{array}{rcl}
	\sigma (f(x)) &=& \sigma ((x-u)~g(x))\\
	&=& \sigma (x-u)~\sigma(g(x))\\
	&=& (\sigma(x)-\sigma(u))~\sigma(g(x))\\
	&=& (x-v)~\sigma(g(x))~~~~~, \mathrm{untuk~suatu~} \sigma(g(x))\in E(v)[x].
	\end{array}$$
\\ 	Berdasarkan hipotesis, $f(x)$ split atas $K$, artinya
	$$\begin{array}{rcl}
	f(x)&=&c(x-u)(x-u_2)...(x-u_n)\\
	&=& (x-u) ~\underbrace{c~(x-u_2)...(x-u_n)}_{g(x)}
	\end{array}$$
\\	Cari lapangan $K$ sedemikian sehingga $\{u~:~f(u)=0\} \subset K$. $F(u)\subseteq K$ maka $K=F(u,u_2,...,u_n)$ dan $g(x)$ split atas $K$.
\\ \\ 	Dapat disimpulkan $K$: lapangan splitting dari $g(x)$ atas $F(u)$ dan $L$: lapangan splitting dari $\sigma g(x)$ atas $E(v)$.
\\ \\ 	Perhatikan bahwa deg$(g(x))=n-1$. Maka berlaku $F(u)\cong E(v).$
\\ \\ 	Jadi, $F(u,u_1,...,u_n)=K\cong L =E(v,v_2,...,v_n)$
	\end{itemize}
	\textbf{Teorema 11.1}
\par 	Misalkan $K,L$ adalah perluasan lapangan dari $F$. $[K:F]< \infty$ dan $[L:F]< \infty$. Jika terdapat isomorphisma $f:K\longrightarrow L$, dengan $\forall c \in F~~f(c)=c$, maka $[K:F]=[L:F]$.
\\
	\textit{Bukti:}
\\	Misal $[K:F]=n$, maka $\{u_1,u_2,...,u_n\}$ merupakan basis untuk $K$ atas $F$. Akan dibuktikan $\{f(u_1),f(u_2),...,f(u_n)\}$ juga merupakan basis dari $L$ atas $F$.\\ \\
	Karena f onto, untuk sembarang $v \in L$, terdapat $u\in K$ sedemikian sehingga $f(u)=v.$ Lebih lanjut, karena $u \in K$, maka $\exists c_i \in F$ sedemikian sehingga $u= \sum^{n}_{i=0} \ c_iu_i$, dan juga $$v=f(u)=f( \sum^{n}_{i=0} \ c_iu_i)= \sum^{n}_{i=0} \ 	f(c_i)f(u_i) =  \sum^{n}_{i=0} \ c_if(u_i).$$
	Artinya, L direntang oleh $\{f(u_1),f(u_2),...,f(u_n)\}$ atau dapat juga ditulis $L=$span$\{f(u_i)\}$. %ada yang harus ditambahin
\\	Selanjutnya, akan ditunjukan bahwa $\{f(u_1),f(u_2),...,f(u_n)\}$ bebas linier.Coba perhatikan,
	$$\begin{array}{rcl}
	0&=&\sum^{n}_{i=0} \ c_if(u_i)\\
	&=& \sum^{n}_{i=0} \ f(c_i)f(u_i)\\
	&=& \sum^{n}_{i=0} \ f(c_iu_i)\\
	&=& f(\sum^{n}_{i=0} \ c_iu_i).\\
	\end{array}$$	
	Karena $f$ adalah pemetaan satu-satu, maka Ker$f=\{0\}$, sehingga $\sum^{n}_{i=0} \ c_iu_i = 0$. Karena $\sum^{n}_{i=0} \ c_iu_i\in K$ dan $\{u_1,u_2,...,u_n\}$ merupakan basis untuk $K$, maka haruslah $c_i=0$ untuk $i=1,2,...,n$. Jadi,
	$\{f(u_1),f(u_2),...,f(u_n)\}$ merupakan basis untuk $L$. Akibatnya $[K:F]=[L:F]$. $\blacksquare$
\\
\\
	\textbf{Definisi 11.1}
\par 	$K$ adalah perluasan aljabar dari $F$. $K$ dikatakan \textbf{normal}, jika suatu polinomial tak tereduksi $f(x)\in F[x]$ punya (minimal satu buah) akar di $K$, maka $f(x)$ \textit{split} atas lapangan $K.$
\\
\\
	\textbf{Teorema 11.2}
\par 	$K$ merupakan \textit{splitting} dari $f(x)\in F[x]$ jika dan hanya jika $K$ normal dan berdimensi hingga.
\\
	\textit{Bukti:}
\\	$\Rightarrow$
\\	Diketahui bahwa $K$ adalah lapangan \textit{splitting} dari $f(x) \in F[x]$, maka $K=F(u_1,u_2,...,u_n)$ di mana $f(u_i)=0~~\forall i$. Berdasarkan Teorema 7.2, $[K:F]< \infty.$
\\	Misal $p(x)\in F[x]$, merupakan polinom tak tereduksi dan $p(v)=0$ untuk suatu $v\in K$, maka $p(x)\in K[x].$
\\	Misalkan $L$ adalah lapangan \textit{splitting} dari $p(x)$ atas $K$, sedemikian sehingga $F\subseteq K \subseteq L.$ Akan ditunjukan bahwa jika $p(u)=0$ dengan $u\in L$, maka $u\in K.$
\\	Misalkan $w\in L$, $w \ne v$ dan $p(w)=0$. Berdasarkan Teorema 10.4, (dalam hal ini $E=F$ dan $\sigma = I$)
	$$\begin{array}{rcl}
	F(v) &\longrightarrow& F(w)\\
	v &\mapsto& w\\
	c &\mapsto& c~~~,\forall c\in F.
	 \end{array}$$
\\	$K(w)\subseteq L$ dan $K(w)=F(w,u_1,u_2,...,u_n) = F(w) (u_1,u_2,...,u_n).$ Artinya, $K(w)$ adalah lapangan \textit{splitting} dari $f(x)$ atas $F(w).$ 
\\ \\ 	
%Ini ga paham lagi
	Ambil sembarang $v\in K$, dengan $K$ adalah lapangan \textit{splitting} dari $f(x)$ atas $F$, dapat pula dinyatakan bahwa $K$ adalah lapangan \textit{splitting} dari $f(x)$ atas $F(v)$. Berdasarkan Teorema 8.2., suatu isomorfisma dari $F(v)$ ke $F(w)$ dapat diekstensi 	ke
	$$\begin{array}{rcl}
	K &\longrightarrow& K(w)\\
	v &\mapsto& w\\
	c &\mapsto& c~~~,\forall c\in F.
	 \end{array}$$
	Berdasarkan Teorema 11.1., $[K:F]=[K(w):F]$.
\\ \\
	Berdasarkan Teorema 6.1., $[K(w):F]=[K(w):K][K:F]$. Akibatnya, $[K:F]=[K(w):K][K:F]$ dan $[K(w):K]=1$. Maka, haruslah $K(w)=K$ dan $w \in K$.
\\ \\
	Sehingga, untuk setiap $u\in L$, sedemikian sehingga $p(u)=0$, maka $u\in K$ dan $p(x)$ split atas $K$. Jadi, K normal.
\\
	$\Leftarrow$
\\ 	Diketahui bahwa $dimK=n<\infty$, maka terdapat $\{e_1,...,e_n\}$ yang merupakan basis untuk $K$. Berdasarkan Teorema 6.3., $F(e_1,...,e_n)= K$ berdimensi hingga. Untuk setiap $i=1,..,n$, $e_i$ adalah aljabar atas F, berdasarkan Teorema 7.1., dengan polinomial-polinomial minimal $P_i(x)\in F[x]$.
\\ \\ 	Selanjutnya, karena $K$ normal, maka setiap  $P_i(x)$ split atas K. Akibatnya, $P_1(x)P_2(x)...P_n(x)=f(x)$ split atas $K$. $\blacksquare$



\chapter{Kenormalan pada Perluasan Lapangan}
	Sebelum memulai \textit{chapter} 11, akan diberikan bukti dari Teorema 8.2, di mana pada \textit{chapter} 8 belum dibuktikan. Selanjutnya, Teorema 8.2 akan digunakan untuk membuktikan beberapa Teorema pada \textit{chapter} ini.
\\ \\	\textit{Bukti: }\textbf{(Teorema 8.2)}\\
	Diketahui bahwa $\sigma : F \longrightarrow E$ adalah suatu isomorfisma, maka 
	$$\begin{array}{rcl}
	\sigma &:& F[x] ~~~~~~~\longrightarrow~~ E[x]\\
	&:&\sum^{n}_{i=0} \ a_ix^i \mapsto \sum^{n}_{i=0} \ \sigma (a_i)x^i=\sigma f(x),
	\end{array}$$ juga suatu isomorphisma.
\\	Selanjutnya, akan digunakan induksi untuk menyelesaikan bukti Teorema ini.
	\begin{itemize}
	\item Untuk deg$(f(x))=1$ dan deg$(\sigma f(x))=1.$
\\	Dikarenakan deg$(f(x))=1$, maka $f(x)=c(x-u)$ dan $u$ merupakan akar dari $f(x)$. $u$ belum tentu elemen di $F$, namun jelas bahwa $u\in F(u)=K.$ Akibatnya, $f(x)\in K[x].$
\\	Perhatikan bahwa, $$f(x)=c(x-u)=cx-cu,$$
	karena $c\in F$ dan $cu\in F$, maka $u\in F$ dan $f(x) \in F[x].$ Sehingga, $K=F(u)=F.$
\\ \\	Selanjutnya, karena $F[x] \cong E[x]$,maka deg$(\sigma f(x))=1$ dan  
	$$\begin{array}{rcl}
	\sigma(f(x))&=&\sigma(c(x-u))\\
	&=&\sigma(cx-cu)\\
	&=&\sigma(cx)-\sigma(cu)\\
	&=&\sigma(c)x-\sigma(c)\sigma(u)\\
	&=&cx-c\sigma(u).
	\end{array}$$
	$\sigma(u)$ belum tentu elemen di $E$, namun jelas bahwa $\sigma(u)\in E(u)=L.$ Akibatnya, $\sigma f(x)\in L[x].$
	Tetapi,karena $c\in E$ dan $c\sigma(u)\in E$, maka $\sigma(u)\in E$ dan $\sigma f(x) \in E[x].$ Sehingga, $L=E(\sigma(u))=E.$
\\	Jadi, $L\cong E \cong F \cong K.$
	\item Asumsikan benar untuk deg$(f(x))=n-1.$
	\item Misal deg$(f(x))=n$.
	%Ada yang ga paham :(
\\	Diketahui, $F[x] \cong E[x]$, maka $\sigma(p(x))$ juga polinomial minimal di $E[x]$, dikarenakan koefisien dari derajat tertinggi $p(x)$ adalah 1 dan $\sigma$ suatu homomorphisma sehingga $\sigma(1)=1$.
\\	Perhatikan bahwa,
	$$f(x)=p(x)\cdot (polinomial-polinomial~tak~tereduksi~yang~lain).$$
	Setiap akar dari $p(x)$, juga merupakan akar dari $f(x)$. Sehingga, akar-akar dari $p(x)\in K.$
\\ \\	Sekarang coba perhatikan,
	$$\sigma f(x)=\sigma p(x)\cdot \sigma((polinomial-polinomial~tak~tereduksi~yang~lain)).$$
	Setiap akar dari $\sigma p(x)$, juga merupakan akar dari $\sigma f(x)$. Sehingga, akar-akar dari $\sigma p(x)\in L.$
\\ \\	Misal $u\in K$ sedemikian hingga $p(u)=0$ dan $v\in L$ sedemikian sehingga $\sigma p(v)=0$. Apakah pemetaan $F(u)\longrightarrow E(v)$ suatu isomorfisma? Ya, Berdasarkan Teorema 10.4
\\ \\	Selanjutnya, karena $p(u)=0$ maka $f(u)=0.$ Sehingga, $f(x)=(x-u)g(x)$ untuk suatu $g(x) \in F(u)[x].$ Diperoleh bahwa
	$$\begin{array}{rcl}
	\sigma (f(x)) &=& \sigma ((x-u)~g(x))\\
	&=& \sigma (x-u)~\sigma(g(x))\\
	&=& (\sigma(x)-\sigma(u))~\sigma(g(x))\\
	&=& (x-v)~\sigma(g(x))~~~~~, \mathrm{untuk~suatu~} \sigma(g(x))\in E(v)[x].
	\end{array}$$
\\ 	Berdasarkan hipotesis, $f(x)$ split atas $K$, artinya
	$$\begin{array}{rcl}
	f(x)&=&c(x-u)(x-u_2)...(x-u_n)\\
	&=& (x-u) ~\underbrace{c~(x-u_2)...(x-u_n)}_{g(x)}
	\end{array}$$
\\	Cari lapangan $K$ sedemikian sehingga $\{u~:~f(u)=0\} \subset K$. $F(u)\subseteq K$ maka $K=F(u,u_2,...,u_n)$ dan $g(x)$ split atas $K$.
\\ \\ 	Dapat disimpulkan $K$: lapangan splitting dari $g(x)$ atas $F(u)$ dan $L$: lapangan splitting dari $\sigma g(x)$ atas $E(v)$.
\\ \\ 	Perhatikan bahwa deg$(g(x))=n-1$. Maka berlaku $F(u)\cong E(v).$
\\ \\ 	Jadi, $F(u,u_1,...,u_n)=K\cong L =E(v,v_2,...,v_n)$
	\end{itemize}
	\textbf{Teorema 11.1}
\par 	Misalkan $K,L$ adalah perluasan lapangan dari $F$. $[K:F]< \infty$ dan $[L:F]< \infty$. Jika terdapat isomorphisma $f:K\longrightarrow L$, dengan $\forall c \in F~~f(c)=c$, maka $[K:F]=[L:F]$.
\\
	\textit{Bukti:}
\\	Misal $[K:F]=n$, maka $\{u_1,u_2,...,u_n\}$ merupakan basis untuk $K$ atas $F$. Akan dibuktikan $\{f(u_1),f(u_2),...,f(u_n)\}$ juga merupakan basis dari $L$ atas $F$.\\ \\
	Karena f onto, untuk sembarang $v \in L$, terdapat $u\in K$ sedemikian sehingga $f(u)=v.$ Lebih lanjut, karena $u \in K$, maka $\exists c_i \in F$ sedemikian sehingga $u= \sum^{n}_{i=0} \ c_iu_i$, dan juga $$v=f(u)=f( \sum^{n}_{i=0} \ c_iu_i)= \sum^{n}_{i=0} \ 	f(c_i)f(u_i) =  \sum^{n}_{i=0} \ c_if(u_i).$$
	Artinya, L direntang oleh $\{f(u_1),f(u_2),...,f(u_n)\}$ atau dapat juga ditulis $L=$span$\{f(u_i)\}$. %ada yang harus ditambahin
\\	Selanjutnya, akan ditunjukan bahwa $\{f(u_1),f(u_2),...,f(u_n)\}$ bebas linier.Coba perhatikan,
	$$\begin{array}{rcl}
	0&=&\sum^{n}_{i=0} \ c_if(u_i)\\
	&=& \sum^{n}_{i=0} \ f(c_i)f(u_i)\\
	&=& \sum^{n}_{i=0} \ f(c_iu_i)\\
	&=& f(\sum^{n}_{i=0} \ c_iu_i).\\
	\end{array}$$	
	Karena $f$ adalah pemetaan satu-satu, maka Ker$f=\{0\}$, sehingga $\sum^{n}_{i=0} \ c_iu_i = 0$. Karena $\sum^{n}_{i=0} \ c_iu_i\in K$ dan $\{u_1,u_2,...,u_n\}$ merupakan basis untuk $K$, maka haruslah $c_i=0$ untuk $i=1,2,...,n$. Jadi,
	$\{f(u_1),f(u_2),...,f(u_n)\}$ merupakan basis untuk $L$. Akibatnya $[K:F]=[L:F]$. $\blacksquare$
\\
\\
	\textbf{Definisi 11.1}
\par 	$K$ adalah perluasan aljabar dari $F$. $K$ dikatakan \textbf{normal}, jika suatu polinomial tak tereduksi $f(x)\in F[x]$ punya (minimal satu buah) akar di $K$, maka $f(x)$ \textit{split} atas lapangan $K.$
\\
\\
	\textbf{Teorema 11.2}
\par 	$K$ merupakan \textit{splitting} dari $f(x)\in F[x]$ jika dan hanya jika $K$ normal dan berdimensi hingga.
\\
	\textit{Bukti:}
\\	$\Rightarrow$
\\	Diketahui bahwa $K$ adalah lapangan \textit{splitting} dari $f(x) \in F[x]$, maka $K=F(u_1,u_2,...,u_n)$ di mana $f(u_i)=0~~\forall i$. Berdasarkan Teorema 7.2, $[K:F]< \infty.$
\\	Misal $p(x)\in F[x]$, merupakan polinom tak tereduksi dan $p(v)=0$ untuk suatu $v\in K$, maka $p(x)\in K[x].$
\\	Misalkan $L$ adalah lapangan \textit{splitting} dari $p(x)$ atas $K$, sedemikian sehingga $F\subseteq K \subseteq L.$ Akan ditunjukan bahwa jika $p(u)=0$ dengan $u\in L$, maka $u\in K.$
\\	Misalkan $w\in L$, $w \ne v$ dan $p(w)=0$. Berdasarkan Teorema 10.4, (dalam hal ini $E=F$ dan $\sigma = I$)
	$$\begin{array}{rcl}
	F(v) &\longrightarrow& F(w)\\
	v &\mapsto& w\\
	c &\mapsto& c~~~,\forall c\in F.
	 \end{array}$$
\\	$K(w)\subseteq L$ dan $K(w)=F(w,u_1,u_2,...,u_n) = F(w) (u_1,u_2,...,u_n).$ Artinya, $K(w)$ adalah lapangan \textit{splitting} dari $f(x)$ atas $F(w).$ 
\\ \\ 	
%Ini ga paham lagi
	Ambil sembarang $v\in K$, dengan $K$ adalah lapangan \textit{splitting} dari $f(x)$ atas $F$, dapat pula dinyatakan bahwa $K$ adalah lapangan \textit{splitting} dari $f(x)$ atas $F(v)$. Berdasarkan Teorema 8.2., suatu isomorfisma dari $F(v)$ ke $F(w)$ dapat diekstensi 	ke
	$$\begin{array}{rcl}
	K &\longrightarrow& K(w)\\
	v &\mapsto& w\\
	c &\mapsto& c~~~,\forall c\in F.
	 \end{array}$$
	Berdasarkan Teorema 11.1., $[K:F]=[K(w):F]$.
\\ \\
	Berdasarkan Teorema 6.1., $[K(w):F]=[K(w):K][K:F]$. Akibatnya, $[K:F]=[K(w):K][K:F]$ dan $[K(w):K]=1$. Maka, haruslah $K(w)=K$ dan $w \in K$.
\\ \\
	Sehingga, untuk setiap $u\in L$, sedemikian sehingga $p(u)=0$, maka $u\in K$ dan $p(x)$ split atas $K$. Jadi, K normal.
\\
	$\Leftarrow$
\\ 	Diketahui bahwa $dimK=n<\infty$, maka terdapat $\{e_1,...,e_n\}$ yang merupakan basis untuk $K$. Berdasarkan Teorema 6.3., $F(e_1,...,e_n)= K$ berdimensi hingga. Untuk setiap $i=1,..,n$, $e_i$ adalah aljabar atas F, berdasarkan Teorema 7.1., dengan polinomial-polinomial minimal $P_i(x)\in F[x]$.
\\ \\ 	Selanjutnya, karena $K$ normal, maka setiap  $P_i(x)$ split atas K. Akibatnya, $P_1(x)P_2(x)...P_n(x)=f(x)$ split atas $K$. $\blacksquare$

\chapter{Grup Galois}
%Pendahuluan%
	Dalam bab ini, akan dikonstruksi Grup Galois dan selanjutnya akan dibahas sifat-sifat yang berkaitan dengan Grup Galois.
\\ \\
	Misalkan $K$ adalah lapangan perluasan atas $F$. $\mathrm{Aut}-F$ dari $K$ didefinisikan sebagai isomorfisma $\sigma: K \longrightarrow K$, sedemikian sehingga $\sigma(c)=c$ untuk setiap $c\in F$. Dengan kata lain, suatu isomorfisma $\sigma: K \longrightarrow K$ merupakan $\mathrm{Aut}-F$ dari $K$, apabila $\sigma$ tersebut direstriksi di $F$ akan sama dengan pemetaan identitas di $F$ i.e. $\sigma |_F=\mathrm{I}_F.$
\\ \\ 
	Selanjutnya didefinisikan himpunan $G$ sebagai berikut,
	$$G:= \{\sigma:K\longrightarrow K | \sigma:\mathrm{Aut}-F~\mathrm{dari}~K\}.$$
	Pada grup ini, didefinisikan operasi komposisi $'\circ'$, yaitu untuk setiap $\sigma,\tau \in G$ dan $x\in K$ berlaku $(\sigma \circ\tau)(x)=\sigma(\tau(x)).$
\\ \\ 
	Perhatikan bahwa,
	\begin{enumerate}
	\item Dapat ditunjukan bahwa $\sigma \circ \tau$ suatu isomorfisma dari $K$ ke $K$. Selanjutnya, untuk setiap $c\in F$ berlaku
	$$(\sigma \circ\tau)(c)=\sigma(\tau(c))=\sigma(c)=c.$$ Sehingga $\sigma \circ \tau\in G$ dan grup $G$ dengan operasi $'\circ'$ bersifat tertutup.
	\item Untuk setiap $\sigma, \tau,\lambda \in G$, berlaku sifat asosiatif, yaitu
	$$(\sigma \circ \tau)\circ \lambda=\sigma \circ (\tau \circ \lambda).$$
	\item Terdapat pemetaan identitas di $K$, yaitu $\iota:K\longrightarrow K$, sedemikian hingga $\iota(c)=(c)$ untuk setiap $c \in K.$ Karenanya,
	$$(\sigma \circ \iota)(x)= \sigma(\iota(x))=\sigma(x),~\mathrm{dan}$$
	$$(\iota \circ \sigma)(x)= \iota(\sigma(x))=\sigma(x).$$
	Selain itu, $\iota$ merupakan isomorfisma dan $\iota |_F=\mathrm{I}_F$, sehingga $\iota \in G.$
	Jadi, $G$ memiliki elemen identitas.
	\item Untuk sembarang $\sigma \in G$, $\sigma$ bersifat bijektif, maka akan terdapat $\sigma^{-1}:K \longrightarrow K$ yang isomorfik sedemikian hingga,
	$$\sigma \circ \sigma^{-1} = \iota = \sigma^{-1} \circ \sigma.$$
	Selanjutnya, untuk setiap $c\in F$
	$$\begin{array}{rcl}
	\sigma(c) &=& c \\
	(\sigma^{-1} \circ \sigma)(c) &=& \sigma^{-1}(c)\\
	\iota (c) &=&  \sigma^{-1}(c)\\
	c &=&  \sigma^{-1}(c)
	\end{array}$$
	Artinya, $ \sigma^{-1}|_F=\mathrm{I}_F$, sehingga,  $\sigma^{-1}\in G$. Jadi, setiap elemen di $G$ memiliki invers. 
	\end{enumerate}
	Maka dari itu, $(G, \circ)$ adalah grup.
\\ \\
	\textbf{Definisi 12.1}
\par 	Misalkan $K$ adalah perluasan lapangan dari $F$. Grup $(G, \circ)$ dengan $G:= \{\sigma:K\longrightarrow K | \sigma:\mathrm{Aut}-F~\mathrm{dari}~K\}$ disebut Grup Galois dari $K$ atas $F$, selanjutnya dinotasikan $\mathrm{Gal}_{F}K$.
\\ \\
	\textbf{Teorema 12.1}
\par 	Misalkan $f(x) \in F[x]$ dan $u$ adalah akar dari f(x). Jika $\sigma \in \mathrm{Gal}_{F}K$ maka $\sigma(u)$ akar dari $f(x).$
\\ \\
	\textit{Bukti:}
\\	Perhatikan bahwa $f(x)=\sum^{n}_{i=0} \ c_ix^i$, dengan $c_i\in F$ untuk setiap $i$. Karena $u$ adalah akar dari f(x), maka $f(u)=\sum^{n}_{i=0} \ c_iu^i=0_F.$
\\ \\ 	Selanjutnya, 
	$$\begin{array}{rcl}
	0_F = \sigma(0_F)&=& \sigma(\sum^{n}_{i=0} \ c_iu^i)\\
	&=& \sum^{n}_{i=0} \ \sigma(c_iu^i)\\
	&=& \sum^{n}_{i=0} \ \sigma(c_i)\sigma(u^i)\\
	&=& \sum^{n}_{i=0} \ \sigma(c_i)\sigma(u)^i\\
	&=& \sum^{n}_{i=0} \ c_i\sigma(u)^i\\
	&=& f(\sigma(u))
	\end{array}$$
	Sehingga, diperoleh $f(\sigma(u))=0_F$. Maka terbukti bahwa $\sigma(u)$ merupakan akar dari $f(x).$ $\blacksquare$
\\ \\ 	Pada teorema di atas, dikatakan bahwa suatu akar dari polinom $f(x)$ akan dipetakan oleh suatu isomorphisma $\sigma \in  \mathrm{Gal}_{F}K$ ke akar dari $f(x)$ tersebut. Untuk mempermudah pemahaman tentang Teorema 12.1, coba perhatikan contoh 		berikut.
\\ \\ 	\textbf{Contoh 1.}
	$\mathbb{C}$ merupakan perluasan lapangan dari $\mathbb{R}$, dengan $\mathbb{C}=\mathbb{R}(\textit{i})=\mathbb{R}(\textit{i,-i})$. Didefinisikan suatu pemetaan yang isomorfisma sebagai berikut,
	$$\begin{array}{rcl}
	\sigma &:& ~~\mathbb{C} ~~~\longrightarrow~~~ \mathbb{C}\\
	&:& \mathrm{a+b}i \mapsto ~\mathrm{a-b}i.
	\end{array}$$
	Perhatikan bahwa, apabila $\sigma$ di restriksi di $\mathbb{R}$, maka untuk setiap $\mathrm{a}\in \mathbb{R}$ berlaku,
	$$\begin{array}{rcl}
	\sigma|_{\mathbb{R}} &:& ~~~\mathbb{C} ~~~~\longrightarrow~~~ \mathbb{C}\\
	&:& \mathrm{a=a+0}i ~\mapsto ~\mathrm{a-0}i=\mathrm{a},
	\end{array}$$
	sehingga $\sigma|_{\mathbb{R}}=\mathrm{I}_{\mathbb{R}}$ dan $\sigma \in \mathrm{Gal}_{\mathbb{R}} \mathbb{C}.$
\\ \\
	Pemetaan dari $i$, dengan $i \notin \mathbb{R}$, yang mungkin adalah
%penomerannya ganti jadi romawi kecil%
	\begin{enumerate}
	\item $i \mapsto i$, yang merupakan pemetaan identitas $\iota \in \mathrm{Gal}_{\mathbb{R}} \mathbb{C}$, dan
	\item $i \mapsto -i$, $\sigma \ne \iota$.
	\end{enumerate}
	
	Jadi, $\mathrm{Gal}_{\mathbb{R}} \mathbb{C} = \{\iota,\sigma\} \cong \mathbb{Z}_2$.
\\ \\	Apakah ada isomorfisma lain selain $\iota$ dan $\sigma$?
\\	Andaikan terdapat $\tau\in \mathrm{Gal}_{\mathbb{R}} \mathbb{C}.$ Ingat bahwa $i$ merupakan akar dari $x^2+1.$ Berdasarkan Teorema 12.1, haruslah $\tau(i)=i$ atau $\tau(i)=-i$, di mana keduanya merupakan akar dari $x^2+1.$ Perhatikan bahwa,
	\begin{itemize}
	\item Jika $\tau(i)=i$, maka $\tau(\mathrm{a+b}i)=\tau(\mathrm{a})+\tau(\mathrm{b})\tau(i)=\mathrm{a+b}i.$ Sehingga, $\tau=\iota.$
	\item Jika $\tau(i)=-i$, maka $\tau(\mathrm{a+b}i)=\tau(\mathrm{a})+\tau(\mathrm{b})\tau(i)=\mathrm{a-b}i.$ Sehingga, $\tau=\sigma.$
	\end{itemize}
	Artinya, jika ada isomorfisma lain, haruslah isomorfisma tersebut sama dengan $\iota$ atau $\sigma$.
\\ \\
	\textbf{Teorema 12.2}
\par 	Misalkan $f(x)\in F[x]$, K : lapangan splitting dari $f(x)$ atas $F$ dan $u,v\in K$. Terdapat $\sigma \in \mathrm{Gal}_{F}K$ sedemikian sehingga $\sigma(u)=v$ jika dan hanya jika $u$ dan $v$ mempunyai polinomial minimal yang sama di $F[x]$.
\\
\par 	Perhatikan kembali Contoh 1, Teorema 12.2 ini, menjamin eksistensi dari $\sigma$ sedemikian hingga $\sigma(i)=-i$, dikarenakan polinomial minimal dari $i$ dan $-i$ sama yaitu $x^2+1.$
\\ \\
	\textbf{Teorema 12.3}
\par 	Misalkan $K=F(u_1,u_2,...,u_n)$ adalah perluasan aljabar dari $F$. Jika $\sigma,\tau \in \mathrm{Gal}_{F}K$ dan $\sigma(u_i)=\tau(u_i)$ untuk setiap $i=1,2,...,n$, maka $\sigma=\tau$.
\\ 
\par 	Biasanya, untuk menunjukan kesamaan dua buah fungsi, harus ditunjukan bahwa semua peta dari domain kedua fungsi tersebut sama. Namun, jika kedua fungsi tersebut merupakan elemen dari suatu Grup Galois atas lapangan $F$, maka berdasarkan Teorema 12.3, cukup 	tunjukan peta dari perluasan lapangan F oleh kedua fungsi tersebut sama.
\\ \\
	\textbf{Definisi 12.2}
\par 	$E$ disebut lapangan tengah dari perluasan $K$ atas $F$, jika $F\subseteq E\subseteq K$.
\\ 
\par 	Perhatikan bahwa, untuk sembarang $\sigma \in \mathrm{Gal}_{E}K$ berlaku $\sigma|_E=\mathrm{I}_E$ dan untuk sembarang $\tau \in \mathrm{Gal}_{F}K$ berlaku $\tau|_F=\mathrm{I}_F.$ Karena $F \subseteq E$, maka $\sigma|_F=\mathrm{I}_F$, namun $\tau|_E   		\ne \mathrm{I}_E.$ Jadi, apabila $F \subseteq E$ maka  $\mathrm{Gal}_{E}K \subseteq  \mathrm{Gal}_{F}K$.
\\ \\
	\textbf{Teorema 12.4}
\par 	Misalkan $K$ adalah perluasan lapangan dari $K$ dan $H$ subgrup dari $\mathrm{Gal}_{F}K$, misalkan $E_H=\{k\in K~:~\sigma(k)=k \mathrm{~untuk~setiap~} \sigma \in H\},$ maka $E_H$ adalah lapangan tengah dari perluasan $K$ atas $F$.
\\ \\
	\textit{Bukti}
\\	Pertama-tama, akan ditunjukan bahwa $E_H$ adalah sublapangan dari $K$.
\\ \\ 	Untuk setiap $c,d\in E_H$ dan untuk setiap $\sigma\in H$, berlaku
	$$\sigma(c+d)=\sigma(c)+\sigma(d)=c+d~~,\mathrm{dan}$$
	$$\sigma(cd)=\sigma(c)\sigma(d)=cd,$$
	dikarenakan $\sigma$ merupakan suatu homomorfisma penjumlahan dan juga perkalian. Maka, $c+d,cd\in E_H$, sama artinya dengan $E_H$ bersifat tertutup terhadap operasi penjumlahan dan perkalian.	
\\ \\ 	Selanjutnya, untuk setiap $\sigma \in H$, berlaku $\sigma(0_F)=0_F$ dan $\sigma(1_F)=1_F$ , maka $0_F,1_F \in E_H.$ Sehingga, $E_H$ memuat elemen identitas penjumlahan dan perkalian.
\\ \\ 	Lalu, untuk sembarang $c\in E_H$, dengan $c \ne 0$, dan untuk setiap $\sigma \in H,$ berlaku
	$$\sigma(-c)=-\sigma(c)=-(c)=-c~~,\mathrm{dan}$$
	$$\sigma(c^{-1})=(\sigma(c))^{-1}=(c)^{-1}=c^{-1}.$$
	Maka, $-c,c^{-1}\in E_H.$ Sehingga, setiap elemen tak nol di $E_H$ memiliki invers penjumlahan dan juga perkalian.
\\ \\ 	Jadi, $E_H$ merupakan lapangan dan $E_H\subseteq K.$
\\ \\ 	Terakhir, untuk setiap $c\in F$ dan $\sigma \in H$, berlaku $\sigma|_F=\mathrm{I}_F~(\sigma(c)=c.)$ Sehingga, $F\subseteq E_H.$
\\ \\ 	Dari uraian di atas diperoleh $F \subseteq E_H\subseteq K$, berdasarkan Definisi 12.2 $E_H$ adalah lapangan tengah dari perluasan $K$ atas $F$.$\blacksquare$ \\

\par Selanjutnya, $E_H$ yang dimaksud pada Teorema 12.4 kita sebut sebagai lapangan \textit{fixed}dari $K$. Berdasarkan teorema di atas,kita dapat mengkonstruksi lapangan tengah dari lapangan \textit{fixed}, namun lapangan tengah tidak harus berasal dari lapangan \textit{fixed}.
\\ \\ 
	\textbf{Contoh 2.} Perhatikan kembali grup $\mathrm{Gal}_{\mathbb{R}} \mathbb{C} = \{\iota,\sigma\}$, seperti yang sudah dipaparkan pada Contoh 1. Subgrup dari $\mathrm{Gal}_{\mathbb{R}} \mathbb{C}$ hanyalah $\{i\}$ dan $\mathrm{Gal}_{\mathbb{R}}  			\mathbb{C}$, maka diperoleh lapangan \textit{fixed} sebagai berikut,
	\begin{itemize}
	\item Lapangan \textit{fixed} dari $\{\iota\}$ adalah $\mathbb{C}$
	\item Lapangan \textit{fixed} dari  $\mathrm{Gal}_{\mathbb{R}} \mathbb{C}$ adalah $\mathbb{R}$.
	\end{itemize}
	Maka, contoh lapangan tengah dari perluasan $\mathbb{C}$ atas $\mathbb{R}$ adalah $\mathbb{C}$ dan $\mathbb{R}$.
\\ \\ 
	\textbf{Contoh 3.} Perhatikan lapangan perluasan dari $\mathbb{Q}$ berikut, yaitu $\mathbb{Q}(\sqrt{3},\sqrt{5}).$ Berdasarkan Teorema 12.1, untuk setiap $\sigma\in \mathrm{Gal}_{\mathbb{Q}}\mathbb{Q}(\sqrt{3},\sqrt{5})$ berlaku
	$$\begin{array}{rcl}
	\sigma&:& \mathbb{Q}(\sqrt{3},\sqrt{5}) \longrightarrow \mathbb{Q}(\sqrt{3},\sqrt{5})\\
		&:& ~~~~~~\sqrt{3}~~~~~\mapsto ~~~~~~\sigma(\sqrt{3})~:~\mathrm{akar~dari~}x^2-3\\
		&~& ~~~~~~\sqrt{5}~~~~~\mapsto ~~~~~~\sigma(\sqrt{5})~:~\mathrm{akar~dari~}x^2-5,\\
	\end{array}$$
	dengan akar dari $x^2-3$ adalah $\pm\sqrt{3}$ dan akar dari $x^2-5$ adalah $\pm\sqrt{5}.$
	Berdasarkan Teorema 12.3, diperoleh empat buah pemetaan yang berbeda yaitu,
	$$\begin{array}{rcl}
	\iota&:& \mathbb{Q}(\sqrt{3},\sqrt{5}) \longrightarrow \mathbb{Q}(\sqrt{3},\sqrt{5})\\
		&:& ~~~~~~\sqrt{3}~~~~~\mapsto ~~~~~~\sqrt{3}\\
		&~& ~~~~~~\sqrt{5}~~~~~\mapsto ~~~~~~\sqrt{5}\\
	\\

	\sigma&:& \mathbb{Q}(\sqrt{3},\sqrt{5}) \longrightarrow \mathbb{Q}(\sqrt{3},\sqrt{5})\\
		&:& ~~~~~~\sqrt{3}~~~~~\mapsto ~~~~~~-\sqrt{3}\\
		&~& ~~~~~~\sqrt{5}~~~~~\mapsto ~~~~~~\sqrt{5}\\
	\\
	
	\alpha&:& \mathbb{Q}(\sqrt{3},\sqrt{5}) \longrightarrow \mathbb{Q}(\sqrt{3},\sqrt{5})\\
		&:& ~~~~~~\sqrt{3}~~~~~\mapsto ~~~~~~\sqrt{3}\\
		&~& ~~~~~~\sqrt{5}~~~~~\mapsto ~~~~~~-\sqrt{5}\\
	\mathrm{dan}\\

	\beta&:& \mathbb{Q}(\sqrt{3},\sqrt{5}) \longrightarrow \mathbb{Q}(\sqrt{3},\sqrt{5})\\
		&:& ~~~~~~\sqrt{3}~~~~~\mapsto ~~~~~~-\sqrt{3}\\
		&~& ~~~~~~\sqrt{5}~~~~~\mapsto ~~~~~~-\sqrt{5}.\\
	\end{array}$$
	$\therefore~\mathrm{Gal}_{\mathbb{Q}}\mathbb{Q}(\sqrt{3},\sqrt{5})=\{\iota,\sigma,\alpha,\beta\}$ dan $\vert\mathrm{Gal}_{\mathbb{Q}}\mathbb{Q}(\sqrt{3},\sqrt{5})\vert=4.$
\\ \\
	Bagaimana mengkonstruksi pemetaan $\sigma,\alpha$ dan $\beta$ yang isomorfisma seperti di atas?
\\ \\	Coba perhatikan cara mengkonstruksi pemetaan $\sigma$ berikut.
\\ \\	Ingat kembali bahwa, $x^2-3$ merupakan polinom minimal dari $\pm\sqrt{3}$ atas $\mathbb{Q}.$ Berdasarkan Teorema 10.4, akan terdapat isomorfisma,
	$$\begin{array}{rcl}
	\tau&:& \mathbb{Q}(\sqrt{3})~~ \longrightarrow \mathbb{Q}~~(-\sqrt{3})\\
		&:& ~~~\sqrt{3}~~~~~\mapsto ~~~~~~-(\sqrt{3})\\
		&~& ~~~~~~c~~~~~\mapsto ~~~~~~c, \mathrm{~untuk~setiap~}c\in \mathbb{Q}.\\
	\end{array}$$
\\ \\	Selanjutnya, $x^2-5$ merupakan polinom minimal dari $\pm\sqrt{5}$ atas $\mathbb{Q}.$ Berdasarkan Teorema 10.4, akan terdapat isomorfisma,
	$$\begin{array}{rcl}
	\bar{\tau}&:&\mathbb{Q}(\sqrt{3},\sqrt{5}) \longrightarrow \mathbb{Q}(\sqrt{3},\sqrt{5})\\
		&:& ~~~~\sqrt{5}~~~~~~~\mapsto ~~~~~~(\sqrt{5})\\
		&~& ~~~~~~~c~~~~~~~\mapsto ~~~~~~c, \mathrm{~untuk~setiap~}c\in \mathbb{Q}.\\
	\end{array}$$
	Maka, $\bar{\tau}=\sigma$Dengan cara yang serupa dapat dikonstruksi $\alpha$ dan $\beta$.
\\ \\	Selanjutnya perhatikan bahwa, 
	\begin{itemize}
	\item $\iota(\sqrt{3})= \sqrt{3},$ sehingga $\vert \iota \vert=1$.
	\item $\sigma \circ \sigma (\sqrt{3})= \sigma(-\sqrt{3})=-\sigma (\sqrt{3})=-(-\sqrt{3})=\sqrt{3},$ sehingga $\vert \sigma \vert=2$.
	\item $\alpha \circ \alpha (\sqrt{5})= \sigma(-\sqrt{5})=-\sigma (\sqrt{5})=-(-\sqrt{5})=\sqrt{5},$ sehingga $\vert \alpha \vert=2$.
	\item $\beta \circ \beta (\sqrt{3})= \beta(-\sqrt{3})=-\beta (\sqrt{3})=-(-\sqrt{3})=\sqrt{3},$ sehingga $\vert \beta \vert=2$.
	\end{itemize}
 	Sekarang, coba perhatikan grup $\mathbb{Z}_2\times \mathbb{Z}_2=\{(0,0),(1,0),(0,1),(1,1)\}$, dengan $\vert (0,0)\vert=1,\vert (1,0)\vert=2,\vert (0,1)\vert=2 \mathrm{~dan~} \vert (1,1)\vert=2.$
\\ \\ 	Karena $\vert\mathrm{Gal}_{\mathbb{Q}}\mathbb{Q}(\sqrt{3},\sqrt{5})\vert=\vert \mathbb{Z}_2\times \mathbb{Z}_2\vert=4$ dan order dari setiap elemennya sama, maka dapat disimpulkan bahwa $\mathrm{Gal}_{\mathbb{Q}}\mathbb{Q}(\sqrt{3},\sqrt{5})\cong \mathbb{Z}_2\times \mathbb{Z}_2.$
\\ \\
	\textbf{Teorema 12.5}
\par 	Jika $K$ adalah lapangan splitting dari $f(x)\in F[x]$, maka $\mathrm{Gal}_FK \cong H$, untuk suatu $H$ subgrup dari grup permutasi.
	


\chapter{Judul Chapter 13}
\textbf{Teorema 12.1}
	\par $K$ lapangan spliting dari $f(x) \in F[x]$, maka $Gal_FK \cong H$ untuk suatu $H$ subgrup dari grup permutasi. Artinya, suatu grup Galois selalu isomorfik dengan subgrup dari grup permutasi.\\
\\ Contoh 1:\\ \\
$\omega=\frac{-1+\sqrt{3}i)}{2}$\\ \\
$\sqrt[3]{2}, \sqrt[3]{2}\omega, \sqrt[3]{2} \omega^2$ merupakan akar-akar dari $x^3-2$.\\ \\
Misal $K$ lapangan spliting dari $x^3-2$ atas $\mathbb{Q}$\\ \\
$Gal_{\mathbb{Q}}K \cong H$, di mana $H$ subgrup dari $S_3$\\
\newpage \textbf{Teorema 12.2}\\ \\
$note:~\sqrt[3]{2}, \sqrt[3]{2}\omega, \sqrt[3]{2} \omega^2~punya~polinomial~minimal~yang~sama.$\\ \\
Maka, $\exists$ isomorfisma $\sigma$ sedemikian sehingga akar dipetakan ke akar. Artinya, jika suatu $a$ dan $b$ merupakan akar dari polinomial minimal yang sama, maka akan selalu ada isomorfisma yang memetakan $a$ ke $b$.\\
Tunjukan pemetaannya apa saja dan bentuk siklik-nya.\\ \\
$$\sqrt[3]{2} \to \sqrt[3]{2}\omega$$
$$\sqrt[3]{2}\omega \to \sqrt[3]{2} \omega^2$$
$$\sqrt[3]{2} \omega^2 \to \sqrt[3]{2}$$\\
$$\begin{pmatrix}
1&2&3\\
2&3&1
\end{pmatrix} (123)$$\\ \\
$x^3-2$ (Merupakan akar dari $\sqrt[3]{2}$\\ \\
$\sqrt[3]{2} \in \mathbb{R}$\\ \\
$\mathbb{Q}(\sqrt[3]{2}) \subset \mathbb{R}$\\ \\
$Gal_{\mathbb{Q}}\mathbb{Q}(\sqrt[3]{2})$\\ \\
$\iota:\sqrt[3]{2} \to \sqrt[3]{2}$\\ \\
$Gal_{\mathbb{Q}}\mathbb{Q}(\sqrt[3]{2})={1} \cong (1)$\\ \\
$\begin{array}{rcl}
E_{Gal_{\mathbb{Q}}\mathbb{Q}(\sqrt[3]{2})}&=&\{k \in \mathbb{Q}(\sqrt[3]{2})|\sigma(k)=k~\forall \sigma \in Gal_{\mathbb{Q}}\mathbb{Q}(\sqrt[3]{2})\} \\
&=&\mathbb{Q}(\sqrt[3]{2})
\end{array}$\\ \\ \\
$\mathbb{Q} \subset \mathbb{Q}(\sqrt[3]{2}) \subset \mathbb{Q}(\sqrt[3]{2},\sqrt[3]{2}\omega,\sqrt[3]{2})$\\ \\
$\begin{array}{rcl}
Misal~\mathcal{S}&=&\{E:lapangan~tengah|F \subset E \subset K\}\\
\mathcal{T}&=&\{H|H~subgrup~dari~Gal_FK\}
\end{array}$\\ \\
Karena $F \subseteq E \subseteq K$, maka $Gal_EK \subseteq Gal_FK$. (karena semua pemetaan di $E$ pasti bisa memetakan semua elemen di $F$ karena $F$ termuat di $E$).\\ \\
Korespondensi Galois dinotasikan dengan $\begin{array}{rcl}
\mathcal{G}&:&\mathcal{S} \to \mathcal{T}\\
& &X \to H_X=Gal_XK
\end{array}$\\ \\
Contoh 2:\\
$\mathbb{Q}(\sqrt{3},\sqrt{5})$ adalah lapangan perluasan dari $\mathbb{Q}$, maka $\mathbb{Q} \subset \mathbb{Q}(\sqrt{3}) \subset \mathbb{Q}(\sqrt{3},\sqrt{5})$.\\
$\mathcal{S}=\{E:lapangan|\mathbb{Q} \subseteq E \subseteq \mathbb{Q}(\sqrt{3},\sqrt{5})\}$\\ \\
$\mathbb{Q} \subseteq, \mathbb{Q}(\sqrt{3}), \mathbb{Q}(\sqrt{3},\sqrt{5}) \in \mathcal{S}$\\ \\
$\begin{array}{rcl}
\mathbb{Q} &\to& H_{\mathbb{Q}}=Gal_{\mathbb{Q}}\mathbb{Q}(\sqrt{3},\sqrt{5})=\{\iota,\sigma,\alpha,\beta\}\\
\mathbb{Q}(\sqrt{3}) &\to& H_{\mathbb{Q}}(\sqrt{3})=Gal_{\mathbb{Q}\sqrt{3}}\mathbb{Q}(\sqrt{3},\sqrt{5})=\{\iota,\alpha\}\\
\mathbb{Q}(\sqrt{3},\sqrt{5}) &\to& H_{\mathbb{Q}}(\sqrt{3},\sqrt{5})=Gal_{\mathbb{Q}\sqrt{3},\sqrt{5}}\mathbb{Q}(\sqrt{3},\sqrt{5})=\{\iota\}
\end{array}$\\ \\
\\ \textbf{Teorema 13.1}
	\par $K$ lapangan perluasan dari $F$, dan $K < \infty$\\
Jika $H$ subgrup dari $Gal_FK$, dan $E$ lapangan fixed dari $H$.\\
maka $H=Gal_EK$ dan $|H|=[K:E]$. ($Gal_EK$ adalah himpunan semua automorfisma yang ketika di-distrik di $E$ merupakan pemetaan identitas).\\ \\
$note:~sebetulnya~bisa~ketika~E~lapangan~tengah.~Tetapi~tidak~istimewa.\\~Namun~ketika~E~lapangan~fixed~, Gal_FK=H.~Menjadikannya~istimewa.$\\ \\
$E$ lapangan fixed $\Rightarrow E$ lapangan tengah.\\
lapangan fixed $\ne \varnothing$.\\ \\
jadi, $\forall H \in \mathcal{T} \Rightarrow \exists E$ lapangan fixied $\in \mathcal{S}$ sedemikian sehingga $\mathcal{G}(E)=Gal_EK=H$ (Teorema 13.1).\\
Artinya $\mathcal{G}$ onto.\\
\\ \textbf{Definisi}
	\par $K$ perluasan lapangan dari $F$, $K$ normal, $K$ separable dan $dim(K)<\infty \iff K$ perluasan Galois dari $F$, atau $K$:Galois atas $F$.\\ \\
\begin{itemize}
\item $f(x) \in F[x];~deg(f(x))=n$ $f(x)$ dikatakan separable jika $|\{u\in K|f(u)=0_F\}|=n$ untuk suatu lapangan spliting $K$ dari $F$. Artinya,suatu polinom dikatakan separable jika banyaknya akar sama dengan derajat-nya.
\item $u \in K$ dikatakan separable atas $F$ jika \begin{enumerate}
\item $u$ aljabar atas $F$.
\item polinom minimal dari $u$ separable.
\end{enumerate}
\item $K$ perluasan separable dari $F$ jika $\forall u \in K$, $u$ separable. Artinya suatu lapangan perluasan dikatakan perluasan separable jika setiap elemennya separable.
\end{itemize}
\textbf{Teorema 13.2}
	\par $K$ perluasan Galois dari $F$, $E$ lapangan tengah\\
Maka $E$ adalah lapangan fixed dari subgrup $Gal_EK$. Artinya jika $K$ perluasan Galois dari $F$, setiap lapangan tengah antara $K$ dan $F$ merupakan lapangan fixed.\\ \\
Semua lapangan tengah bisa menjadi lapangan fixed, asalkan $K$ perluasan Galois dari $F$.\\ \\
Misal $E,L$ lapangan tengah dari perluasan $K$ atas $F$ dengan $Gal_EK=Gal_LK$.\\
Jika $K$ perluasan Galois dari $F$, maka berdasarkan Teorema 13.2.\\
$E=L$.\\ \\
Jika $K$ perluasan Galois. $\mathcal{G}$ menjadi bijektif. Jika bukan, maka $\mathcal{G}$ hanya onto. 


\chapter{Ujian Akhir Semester}
\end{document}